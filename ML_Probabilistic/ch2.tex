\section{Probability}

\exercise[]{
\textbf{
My neighbor has two children. Assuming that the gender of a child is
like a coin flip, it is most likely, a priori, that my neighbor has one boy
and one girl, with probability 1/2. The other possibilities, two boys
or two girls, have probabilities 1/4 and 1/4. \\ \\
a. Suppose I ask him whether he has any boys, and he says yes. What is the
probability that one child is a girl? \\ \\
b. Suppose instead that I happen to see one of his children run by, and it is
a boy. What is the probability that the other child is a girl?}

a. Let $G$ represent one girl, and $B$ represent one boy. Since the neighbor
has two children, we can state the entire sample space:

$$S = \{BB, BG, GB, GG\}$$

When the neighbor answers the question, this changes our beliefs about the
other child. Using Bayes' theorem:

\begin{align}
    P(G = 1 | B \geq 1) & = \frac{P(B \geq 1 | G = 1) P(G = 1)}{P(B \geq 1)} \\
                        & = \frac{2/2 \times 1/2}{3/4} \\
                        & = \frac{2}{3}
\end{align}

b. If we instead happen to see one of his children, this is a different way
of looking at the problem. In this situation, learning the gender of one child
tells us nothing about the gender of the other child. Therefore, the gender of
the second child is a coin flip, $1/2$.

}

\exercise[]{
\textbf{
Suppose a crime has been committed. Blood is found at the scene for which there
is no innocent explanation. It is of a type which is present in 1\% of the
population. \\ \\
a. The prosecutor claims: "There is a 1\% chance that the defendant would have
the crime blood type if he were innocent. Thus there is a 99\% chance that he
is guilty". This is known as the prosecutor's fallacy. What is wrong with this
argument? \\ \\
b. The defender claims: "The crime occurred in a city of 800,000 people. The
blood type would be found in approximately 8000 people. The evidence has
provided a probability of just 1 in 8000 that the defendant is guilty, and
thus has no relevance". This is known as the defender's fallacy. What is wrong
with this argument?}

a. The defendant sharing the blood type does not mean that the defendant
himself has a 99\% probability of being guilty, just that he shares the same
blood type as the guilty party, just like he shares the same blood type with
1\% of the population. In a large enough city, there would be a large number
of people fitting this description in a small geographical radius.

b. This statement assumes that the defendent is just as guilty (or just as
non-guilty) as anyone else in that group of 8000 people. If there truly is
no other evidence to tie this defendent to this crime, then that may be so,
but if there were any other evidence (drives a similar car as the criminal,
lives in the same area, or frequents the same locations), the probability
that the defendant is guilty could be much higher.
}

\exercise[]{
\textbf{Show that the variance of a sum is $Var[X + Y] = Var[X] + Var[Y] + 2Cov[X,Y]$,
where $Cov[X, Y]$ is the covariance between $X$ and $Y$.}

\begin{align}
    & Var[X] + Var[Y] + 2Cov[X, Y] = E[(X - \mu_x)^2] + E[(Y - \mu_y)^2] + 2E[(X - \mu_x)(Y - \mu_y)] \\
    & = E[X^2 - 2X\mu_x + \mu_x^2] + E[Y^2 - 2Y\mu_y + \mu_y^2] + E[2XY - 2X\mu_y - 2Y\mu_x + 2\mu_x\mu_y] \\
    & = E[X^2 - 2X\mu_x + \mu_x^2 + Y^2 - 2Y\mu_y + \mu_y^2 + 2XY - 2X\mu_y - 2Y\mu_x + 2\mu_x\mu_y] \\
    & = E[X^2 + 2XY - 2X(\mu_x + \mu_y) + Y^2 - 2Y(\mu_x + \mu_y) + 2\mu_x\mu_y]
\end{align}

Note that $E(X + Y) = E(X) + E(Y) = \mu_x + \mu_y = \mu_{xy}$. Given this,

\begin{align}
    & E[X^2 + 2XY - 2X(\mu_x + \mu_y) + Y^2 - 2Y(\mu_x + \mu_y) + 2\mu_x\mu_y] \\
    & = E[X^2 + 2XY - 2X\mu_{xy} + Y^2 - 2Y\mu_{xy} + 2\mu_x\mu_y] \\
    & = E[(X + Y - \mu_{xy})^2] \\
    & = Var[X + Y]
\end{align}

}

\exercise[]{
\textbf{After your yearly checkup, the doctor has bad news and good news. The
bad news is that you tested positive for a serious disease, and that the test
is 99\% accurate (i.e., the probability of testing positive given that you have
the disease is 0.99, as is the probability of testing negative given that you
don't have the disease). The good news is that this is a rare disease, striking
only one in 10,000 people. What are the chances that you actually have the
disease? (Show your calculations as well as giving the final result.)}

Since the test is 99\% accurate, we know that $P(Y | D) = P(N | ~D) = 0.99$,
where $Y$ means a positive test result, $N$ a negative test result, $D$ means
you have the disease, and $~D$ means you do not have the disease. We also know
the prior probability of having the disease: $P(D) = 0.0001$. Using Bayes'
rule:

\begin{align}
    P(D | Y) & = \frac{P(Y | D)P(D)}{P(Y)} \\
             & = \frac{0.99 \times 0.0001}{P(Y | D)P(D) + P(Y | ~D)P(~D)} \\
             & = \frac{0.000099}{(0.99 \times 0.0001) + (0.01 \times 0.9999)} \\
             & = \frac{0.000099}{0.000099 + 0.009999} \\
             & = 0.0098
\end{align}

So, there's about a 1\% chance that you have the disease even though you tested
positive for it.

\exerciseshere
