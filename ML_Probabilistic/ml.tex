\documentclass{article}
\usepackage{amsthm, amssymb, amsmath, tikz}

\theoremstyle{remark}
\newtheorem{ExInternal}{Exercise}[section]

\makeatletter
\let\@exercises\@empty%
\newcommand\exercise[2][]{%
    \g@addto@macro\@exercises{%
        \begin{ExInternal}[#1]%
            #2%
        \end{ExInternal}%
    }%
}

\newcommand\exerciseshere{%
    \subsection*{Exercises}
    \@exercises%
    \global\let\@exercises\@empty%
}

\makeatother
\begin{document}

\title{Selected Exercises of Machine Learning: A Probabilistic Perspective}
\author{Mike Craig}
\date{Last Updated \today}

\maketitle

\section{Introduction}

This is a collection of certain exercise solutions I did as I was reading
through this book. As I am not a mathematician, the "proofs" listed here
are rough and probably not rigorous. They only serve as a means to explain
the concepts in a way that aids my own understanding.

\section{Overview of Supervised Learning}

\exercise[]{
\textbf{Suppose each of $K$-classes has an associated target $t_k$, which is a
vector of all zeros, except a one in the $k$th position. Show that classifying
to the largest element of $\hat{y}$ amounts to choosing the closest target,
$min_k ||t_k - \hat{y}||$, if the elements of $\hat{y}$ sum to one.}

We are trying to show that

$$arg\,max_k\,\hat{y_k} = arg\,min_k ||t_k - \hat{y}||$$

To do this, we can show that for any $k^* \neq arg\,max_k\,\hat{y_k}$
and $k = arg\,max_k\,\hat{y_k}$,

$$||t_{k^*} - \hat{y}|| > ||t_{k} - \hat{y}||$$

Note that we can equivalently consider the $||\cdot||^2$ instead of $||\cdot||$,
because they are both monotonic for $\geq 0$.

Using the definition of the Euclidean norm,

\begin{align}
||t_{k^*} - \hat{y}||^2& = ||\hat{y}||^2 + ||t_{k^*}||^2 - 2t_{k^*}\hat{y} \\
                       & = \hat{y}^2 \\
\end{align}

Similarly,

\begin{align}
||t_k - \hat{y}||^2& = ||\hat{y}||^2 + ||t_k||^2 - 2t_k\hat{y} \\
                   & = \hat{y}^2 + 1 - 2\hat{y}
\end{align}

Therefore

\begin{align}
||t_{k^*} - \hat{y}||^2 - ||t_k - \hat{y}||^2& = \hat{y}^2 - (\hat{y}^2 + 1 - 2\hat{y}) \\
                 & = -1 + 2\hat{y} \\
                 & \geq 0
\end{align}

since $\hat{y}$ sums to one. So, $||t_k - \hat{y}||$ is minimized when
$k = arg\,max_k\,\hat{y_k}$.

}

\exercise[]{
Show how to compute the Bayes decision boundary for the simulation example
in Figure 2.5.
}

\exercise[]{
\textbf{Consider N data points uniformly distributed in a p-dimensional unit
ball centered at the origin. Suppose we consider a nearest-neighbor estimate
at the origin. Show that the median distance from the origin to the closest
data point is given by: $d(p, N) = (1 - \frac{1}{2}^{1/N})^{1/p}$.}

Let $m$ be the median distance from the origin to the closest point. This
means that the probability that all data points are further than $m$ is $0.5$.
"Further" simply means a greater norm. Since samples $x_i$ are i.i.d, we can
more formally state this as:

$$\prod_{i=1}^{N} P(||x_i|| > m) = \frac{1}{2}$$

Note that we can flip this around to be
$\prod_{i=1}^{N} P(||x_i|| \leq m) = \frac{1}{2}$.
Now we can use the cumulative function of the uniform distribution as follows:

\begin{align}
\prod_{i=1}^{N} P(||x_i|| \leq m)& = \prod_{i=1}^{N} 1 - ||m|| \\
                                 & = \prod_{i=1}^{N} 1 - m^p \\
                                 & = (1 - m^p)^N = \frac{1}{2}
\end{align}

Now we can solve for $m$:

\begin{align}
\frac{1}{2} & = (1 - m^p)^N \\
\frac{1}{2}^{1/N} & = 1 - m^p \\
m^p & = 1 - \frac{1}{2}^{1/N} \\
m & = (1 - \frac{1}{2}^{1/N})^{1/p}
\end{align}

}

\exercise[]{
\textbf{The edge effect problem discussed on page 23 is not peculiar to uniform
sampling from bounded domains. Consider inputs drawn from a spherical
multinormal distribution $X ∼ N(0, I_p)$. The squared distance from any sample
point to the origin has a $\chi^2_p$ distribution with mean $p$. Consider a
prediction point $x_0$ drawn from this distribution, and let $a = x_0/||x_0||$
be an associated unit vector. Let $z_i = a^Tx_i$ be the projection of each of
the training points on this direction. Show that the $z_i$ are distributed
$N(0, 1)$ with expected squared distance from the origin 1, while the target
point has expected squared distance $p$ from the origin. Hence for $p = 10$,
a randomly drawn test point is about 3.1 standard deviations from the origin,
while all the training points are on average one standard deviation along
direction $a$. So most prediction points see themselves as lying on the edge
of the training set.}

}

\exercise[]{
\textbf{(a) Derive equation (2.27)}

(a) Equation (2.27) states

\begin{align}
EPE(x_0) & = E_{y_0|x_0} E_{\tau} (y_0 - \hat{y}_0)^2 \\
         & = Var(y_0 | x_0) + E_{\tau} [\hat{y}_0 - E_{\tau} \hat{y}_0]^2 + [E_{\tau} \hat{y}_0 - x_0^T\beta]^2 \\
         & = Var(y_0 | x_0) + Var_{\tau}(\hat{y}_0) + Bias^2(\hat{y}_0) \\
         & = \sigma^2 + E_{\tau} x_0^T (\mathbf{X}^T\mathbf{X})^{-1} x_0\sigma^2 + 0^2.
\end{align}

We can derive this is in the three constituent parts shown. Let's start with
the bias squared term. Since we know that the least squares estimator is an
unbiased estimator with respect to the training set, it is obvious that

$$[E_{\tau} \hat{y}_0 - x_0^T\beta]^2 = [x_0^T\beta - x_0^T\beta]^2 = 0^2$$

So, the bias term is zero. Next, we will look at the middle term. Note that
for the least squares solution $\hat{\beta}$,
$Var(\hat{\beta}) = (\mathbf{X}^T\mathbf{X})^{-1} \sigma^2$. So,

\begin{align}
Var_{\tau}(\hat{y}_0) & = Var_{\tau}(x_0^T\hat{\beta}) \\
     & = x_0^2Var(\hat{\beta}) \\
     & = E_{\tau} x_0^T (\mathbf{X}^T\mathbf{X})^{-1} x_0\sigma^2
\end{align}

Finally, it is clear that $Var(y_0 | x_0) = \sigma^2$, since this is how we
define $\sigma^2$.

(b) Equation (2.28) states

\begin{align}
E_{x_0}EPE(x_0) & \sim E_{x_0} x_0^TCov(X)^{-1}x_0\sigma^2 / N + \sigma^2 \\
                & = trace[Cov(X)^{-1}Cov(x_0)]\sigma^2 / N + \sigma^2 \\
                & = \sigma^2(p / N) + \sigma^2
\end{align}

}

\exerciseshere

\section{Generative Models for Discrete Data}

\exercise[]{
\textbf{
Derive Equation 3.22 by optimizing the log likelihood in Equation 3.11}

Equation 3.22 states $\hat{\theta}_{MLE} = \frac{N_1}{N}$.

To derive this, we must maximize the log likelihood. Formally, we have
to find

\begin{align}
    argmax_{\theta}\, p(D|\theta) & = \theta^{N_1}(1 - \theta)^{N_0} \\
    argmax_{\theta}\, log\,p(D|\theta) & = N_1log(\theta)+N_0log(1 - \theta) \\
    argmin_{\theta}\,-log\,p(D|\theta) & = -N_0log(1 - \theta)-N_1log(\theta)
\end{align}

We will minimize this by taking the derivative equal to $0$ and solving
for $\theta$:

\begin{align}
    0 & = \frac{d}{d\theta} -N_0log(1 - \theta) - N_1log(\theta) \\
      & = \frac{N_0}{1 - \theta} - \frac{N_1}{\theta} \\
    \frac{N_1}{\theta} & = \frac{N_0}{1 - \theta} \\
    N_1(1 - \theta) & = N_0\theta \\
    N_1 - N_1\theta & = N_0\theta \\
    N_1 & = (N_0 + N_1)\theta \\
    \frac{N_1}{N} & = \theta
\end{align}

}

\exercise[]{
\textbf{Derive the following:}

\begin{align}
    p(D) & = \frac{[(\alpha_1)\cdots (\alpha_1+N_1-1)]
                   [(\alpha_0)\cdots (\alpha_0+N_0-1)]}
                  {(\alpha)\cdots (\alpha + N - 1)} \\
         & = \frac{\Gamma(\alpha_1+N_1)\Gamma(\alpha_0+N_0)}
                  {\Gamma(\alpha_1 + \alpha_0 + N)}
             \frac{\Gamma(\alpha_1 + \alpha_0)}{\Gamma(\alpha_1)\Gamma(\alpha_0)}
\end{align}

To derive this, we must use the identity $\Gamma(\alpha) = (\alpha - 1)!$. Also
note that

$$\frac{\Gamma(\alpha + k)}{\Gamma(\alpha)} = (\alpha) \cdots (\alpha + k)$$

Using these, and the fact that $\alpha = \alpha_0 + \alpha_1$,

\begin{align}
    p(D) & = \frac{[(\alpha_1)\cdots (\alpha_1+N_1-1)]
                   [(\alpha_0)\cdots (\alpha_0+N_0-1)]}
                  {(\alpha)\cdots (\alpha + N - 1)} \\
         & = \frac{(\Gamma(\alpha_1+N_1)/\Gamma(\alpha_1))
                   (\Gamma(\alpha_0+N_0)/\Gamma(\alpha_0))}
                  {\Gamma(\alpha+N)/\Gamma(\alpha)} \\
         & = \frac{\Gamma(\alpha_1+N_1)\Gamma(\alpha_0+N_0) /
                   \Gamma(\alpha_1)\Gamma(\alpha_0)}
                  {\Gamma(\alpha+N)/\Gamma(\alpha)} \\
         & = \frac{\Gamma(\alpha_1+N_1)\Gamma(\alpha_0+N_0)\Gamma(\alpha)}
                  {\Gamma(\alpha+N)\Gamma(\alpha_1)\Gamma(\alpha_0)} \\
         & = \frac{\Gamma(\alpha_1+N_1)\Gamma(\alpha_0+N_0)}{\Gamma(\alpha+N)}
             \frac{\Gamma(\alpha)}{\Gamma(\alpha_1)\Gamma(\alpha_0)} \\
         & = \frac{\Gamma(\alpha_1+N_1)\Gamma(\alpha_0+N_0)}
                  {\Gamma(\alpha_0 + \alpha_1 +N)}
             \frac{\Gamma(\alpha_0 + \alpha_1)}
                  {\Gamma(\alpha_1)\Gamma(\alpha_0)}
\end{align}

}

\exercise[]{
\textbf{Show that}

$$p(x|n,D) = \binom{n}{x} \frac{B(x + \alpha'_1, n - x + \alpha'_0)}
                               {B(\alpha'_1, \alpha'_0)}$$

\textbf{reduces to $p(x=1|D) = \frac{\alpha'_1}{\alpha'_0 + \alpha'_1}$
when $n = 1$.}

Let $n = 1$, and $x \in \{0, 1\}$. The Beta-Binomial model is given by:

$$Bb(x|a,b,n) = \binom{n}{x} \frac{B(x+a,n-x+b)}{B(a,b)}$$

Plugging what we know in,

\begin{align}
    Bb(1|\alpha'_1, \alpha'_0, 1) & = \binom{1}{1}
                                      \frac{B(1+\alpha'_1,1-1+\alpha'_0)}
                                           {B(\alpha'_1,\alpha'_0)} \\
    & = \frac{\Gamma(1+\alpha'_1)\Gamma(\alpha'_0)/\Gamma(1+\alpha'_1+\alpha'_0)}
             {\Gamma(\alpha'_1)\Gamma(\alpha'_0)/\Gamma(\alpha'_1+\alpha'_0)} \\
    & = \frac{\Gamma(1+\alpha'_1)\Gamma(\alpha'_0)\Gamma(\alpha'_1+\alpha'_0)}
             {\Gamma(1+\alpha'_1+\alpha'_0)\Gamma(\alpha'_0)\Gamma(\alpha'_1)} \\
    & = \frac{\Gamma(1+\alpha'_1)}{(\alpha'_0+\alpha'_1+1)\Gamma(\alpha'_1)} \\
    & = \frac{(\alpha'_1+1)\Gamma(\alpha'_1)}
             {(\alpha'_0+\alpha'_1+1)\Gamma(\alpha'_1)} \\
    & = \frac{\alpha'_1+1}{\alpha'_0+\alpha'_1+1}
\end{align}

}

\exercise[]{
\textbf{Suppose we toss a coin $n=5$ times. Let $X$ be the number of heads.
Let the prior probability of heads be $p(\theta)=Beta(\theta|1,1)$. Compute
the posterior $p(\theta|X<3)$ up to normalization constant.}

The posterior can be given by $p(\theta|D) = p(D|\theta)p(\theta)$. By plugging
in the likelihood (Binomial) and the prior (Beta), we get

\begin{align}
    p(\theta|D) & = p(D|\theta)p(\theta) \\
                & \propto Beta(\theta | N_1 + \alpha, N_2 + \beta)
\end{align}

Since the number of heads is discrete and mutually exclusive,

\begin{align}
    p(\theta|X<3) & = p(\theta|X=0) + p(\theta|X=1) + p(\theta|X=2) \\
                  & \propto Beta(\theta | 1, 5) +
                            Beta(\theta | 2, 4) +
                            Beta(\theta | 3, 3) \\
                  & = \sum_{a=0}^{2} Beta(\theta | a + 1, 5 + 1 - a)
\end{align}

}

\exercise[]{
\textbf{Let $\phi = logit(\theta) = log\frac{\theta}{1-\theta}$. Show that if
$p(\phi) \propto 1$, then $p(\theta) \propto Beta(\theta|0,0)$.}

Using the change of variables formula,

\begin{align}
    p(\phi) & = p(logit(\theta)) \\
    & = \left |\frac{d\phi}{d\theta}\right | p(log\frac{\theta}{1-\theta}) \\
    & \propto \left | \frac{d}{d\theta} \left (\frac{\theta}{1-\theta}\right ) \right | \\
    & = \theta^{-1}(1-\theta)^{-1}
\end{align}

Note that a $Beta(\theta|0,0)$ distribution can be defined as

\begin{align}
    Beta(\theta|0,0) & = \frac{\theta^{0-1}(1-\theta)^{0-1}}{B(0,0)} \\
                     & \propto \theta^{-1}(1-\theta)^{-1}
\end{align}

This result is important, because it shows us that the uniform distribution
transformed using the logit function is a $Beta(0,0)$ distribution, which
is an uninformative Beta distribution.

}

\exercise[]{
\textbf{The Poisson pmf is defined as
$Poi(x|\lambda) = e^{-\lambda}\frac{\lambda^x}{x!}$, for which
$x \in \{ 0, 1, 2, ... \}$ where $\lambda > 0$ is the rate parameter.
Derive the MLE.}

The MLE is defined as

\begin{align}
    argmax_{\lambda} p(x|\lambda) & = e^{-\lambda}\frac{\lambda^x}{x!} \\
    argmax_{\lambda} log\,p(x|\lambda) & = -\lambda + xlog\lambda -logx! \\
    argmin_{\lambda} -log\,p(x|\lambda) & = \lambda - xlog\lambda + logx!
\end{align}

By taking the derivate and setting it to $0$,

\begin{align}
    0 & = \frac{d}{d\lambda} \left | \lambda - xlog\lambda + logx! \right | \\
      & = 1 - \frac{x}{\lambda} \\
    \frac{x}{\lambda} & = 1 \\
    x & = \lambda
\end{align}

Therefore, the MLE solution to the Poisson is $x = \lambda$.

}

\exercise[]{
\textbf{a. Derive the posterior $p(\lambda|D)$ assuming a conjugate prior
$p(\lambda) = Ga(\lambda|a,b) \propto \lambda^{a-1}e^{-\lambda b}$.}

The posterior is given by

\begin{align}
    p(\lambda|D) & = p(D|\lambda)p(\lambda) \\
    & \propto e^{-\lambda}\frac{\lambda^x}{x!}\lambda^{a-1}e^{-\lambda b} \\
    & \propto \lambda^x\lambda^{a-1}e^{-\lambda}e^{-\lambda b} \\
    & = \lambda^{x+a-1}e^{-\lambda(b + 1)} \\
    & = Ga(a + x, b + 1)
\end{align}

\textbf{b. What does the posterior mean tend to as $a \rightarrow 0$ and
$b \rightarrow 0$? (Recall that the mean of a $Ga(a,b)$ distribution is $a/b$.}

The posterior mean tends to $Ga(0 + x, 0 + 1) = Ga(x, 1) \rightarrow x$ when
$a \rightarrow 0$ and $b \rightarrow 0$.

}

\exercise[]{
\textbf{Consider a uniform distribution centered on $0$ with width $2a$. The
density function is given by}

$$p(x) = \frac{1}{2a} I(x \in [-a, a])$$

\textbf{a. Given a data set $x_1, ..., x_n$, what is the maximum likelihood
estimate of $a$ (call it $\hat{a}$)?}

\begin{align}
    p(D|a) & = \prod_{i=1}^{n} p(x_i|a) \\
           & = \prod_{i=1}^{n} \frac{1}{2a} I(x \in [-a, a]) \\
           & = \frac{1}{(2a)^n} \prod_{i=1}^{n} I(x \in [-a, a])
\end{align}

Since, this is the quantity we want to maximize. Note that it is maximized
as $a$ is minimal (first term). The second term nullifies the first term for
all $x_i$ that is outside the interval $[-a, a]$. This means that the posterior
is maximized for the smallest interval $[-a, a]$ that captures the full range
of the data. Formally, the posterior is maximized when

$$\hat{a} = max(|x_i|)$$

\textbf{b. What probability would the model assign a new data point $x_{n+1}$
using $\hat{a}$?}

Since $p(x_{n+1}) = \frac{1}{2\hat{a}} I(x_{n+1} \in [-\hat{a}, \hat{a}])$,
it is obvious that $x_{n+1}$ has probability $\frac{1}{2\hat{a}}$ if the
point $x_{n+1}$ is in the range $[-\hat{a}, \hat{a}]$, and $0$ otherwise.

\textbf{c. Do you see any problem with the above approach? Briefly suggest
(in words) a better approach.}

This approach suffers from the zero-count problem. In general, any probability
specification that assigns zero probability to inputs that are possible is not
ideal. A better solution would to use some Bayesian approach, or add Laplace
smoothing.

}

\exercise[]{
\textbf{Derive the posterior $p(\theta|D)$ of the uniform with a Pareto prior,
and show that it can be expressed as a Pareto distribution.}

Note that the Pareto distribution is given by

$$Pareto(\theta|b,K) = bK^b\theta^{-(b+1)}I(\theta \geq K)$$

Using this,

\begin{align}
    p(\theta|D) & = \frac{p(D,\theta)}{p(D)} \\
    & = \frac{\frac{Kb^K}{\theta^{N+K+1}}}
             {\int_m^{\infty} \frac{Kb^K}{\theta^{N+K+1}}d\theta}
        I(\theta \geq max(D)) \\
    & = \left\{\begin{matrix}
        \frac{K\theta^{N+K-1}}{K(N+K)b^{N+K}}I(\theta \geq m) & if\,m \leq b\\ 
        \frac{K\theta^{N+K-1}}{Kb^K(N+K)m^{N+K}}I(\theta \geq m) & if\,m > b
        \end{matrix}\right. \\
    & = \frac{\theta^{N+K-1}I(\theta \geq m)}{N+K} \left\{\begin{matrix}
        b^{-N-K} & if\,m \leq b\\ 
        b^{-K}m^{-K-N} & if\,m > b
        \end{matrix}\right. \\
    & \propto \theta^{N+K-1}b^{-K}m^{-K-N}I(\theta \geq m) \\
    & = Pareto(\theta | -(K+N), m)
\end{align}

}

\exercise[]{
\textbf{Let's say that taxicars are numbered uniformly like
$p(x) = U(0,\theta)$. \\
a. Suppose we see one taxi numbered $100$, so $D = \{ 100 \}$, $m=100$,
$N=1$. Using a non-informative prior on $\theta$ of the form
$p(\theta) = Pa(\theta|0,0) \propto 1/\theta$, what is the posterior
$p(\theta|D)$?}

Recall from the previous exercise that the posterior is a Pareto
of the form $Pa(\theta|N+K,max(m,b))$. The posterior is then given by

\begin{align}
    p(\theta|D) & = p(D|\theta)p(\theta) \\
                & = U(0, \theta)Pa(\theta|0,0) \\
                & = Pa(\theta|N+0,max(100,0)) \\
                & = Pa(\theta|1,100)
\end{align}

\textbf{b. Compute the posterior mean, mode and median number of taxis
in the city, if such quantities exist.}

We know the form of the posterior, so the posterior mean is the mean of
the Pareto distribution, which is given by

$$\mu_{a,b} = \frac{ab}{a - 1}$$

therefore, the mean of $Pa(\theta|1,100)$ is $\frac{100}{0}$, which is
undefined.

The mode of a $Pa(\theta|a,b)$ is $b$, so the mode of the posterior
is $m = 100$.

The median of a $Pa(\theta|a,b)$ is $2^{1/a}b$, so the mode of the posterior
is $2^{1/1}\times 100 = 200$.

\textbf{c. Compute the predictive density for the next taxicab number.}

We can use the above equations to find the prior before witnessing the
second taxicab. This prior will be the posterior after seeing the first
taxicab number. This posterior is given by $Pa(\theta|1, m)$. Thus, using
$b=m$ and $K=1$, we can plug this into the equation above as

\begin{align}
    p(x|D,K,b) & = \frac{K}{(N+K)b^N} I(x \leq m) + \frac{Kb^K}{(N+K)m^{N+K}} I(x > m) \\
    & = \frac{1}{(1+1)m^1} I(x \leq m) + \frac{m^1}{(1+1)x^{1+1}} I(x > m) \\
    & = \frac{1}{2m} I(x \leq m) + \frac{m}{2x^2} I(x > m)
\end{align}

\textbf{d. Use the predictive density to compute the probability that the
next taxi you will see (say, the next day) has number 100, 50, or 150, i.e.
compute $p(x = 100|D,\alpha)$, etc.}

\begin{align}
    p(x=100|D,\alpha) & = \frac{1}{2m} I(x\leq m) + \frac{m}{20000} I(x>m) \\
    p(x=50|d,\alpha) & = \frac{1}{2m} I(x \leq m) + \frac{m}{5000} I(x>m) \\
    p(x=150|d,\alpha) & = \frac{1}{2m}I(x \leq m) + \frac{m}{45000} I(x>m)
\end{align}

\textbf{e. Briefly describe some ways we might make the model more accurate
at prediction.}

We are currently using an uninformative prior, which doesn't seem ideal. There
are certain restrictions we could make on the distribution of taxi numbers.

}

\exercise[]{
\textbf{The exponential distribution with parameter $\theta$ is given by
$p(x|\theta) = \theta e^{-\theta x}$. \\
a. Show that the MLE is given by $\hat{\theta} = 1/\bar{x}$, where
$\bar{x} = \frac{1}{N} \sum_{i=1}^{N} x_i$.}

The log likelihood is given by

\begin{align}
    log\,p(x|\theta) & = \sum_{i=1}^{N} log(\theta e^{-\theta x_i}) \\
                     & = \sum_{i=1}^{N} log(\theta) - \theta x_i \\
                     & = Nlog(\theta) - \sum_{i=1}^{N} \theta x_i \\
                     & = Nlog(\theta) - \theta \sum_{i=1}^{N} x_i
\end{align}

Setting the derivative to $0$,

\begin{align}
    0 & = \frac{d}{d\theta}\left |Nlog(\theta)-\theta \sum_{i=1}^{N}x_i\right | \\
    & = \frac{N}{\theta} - \sum_{i=1}^{N} x_i \\
    \sum_{i=1}^{N} x_i & = \frac{N}{\theta} \\
    \theta & = \frac{N}{\sum_{i=1}^{N} x_i} \\
           & = \frac{1}{\bar{x}}
\end{align}

\textbf{b. Suppose we observe $X_1 = 5$, $X_2 = 6$, $X_3 = 4$. What is the
MLE given this data?}

The MLE is one over the arithmetic mean, which is $1 / mean(5, 6, 4) = 1/5$.

\textbf{c. Assume that an expert believe $\theta$ should have a prior
distribution that is also exponential $p(\theta) = Expon(\theta|\lambda)$.
Choose the prior parameter, call it $\hat{\lambda}$, such that $E[\theta] = 1/3$.}

Note that the exponential distribution is just a special case of the Gamma
distribution. In particular, $Expon(x|\theta) = Gamma(x|1,1/\theta)$. Since
we know that the mean of the Gamma distribution is $a/b$, then we can find
the exponential with mean of $1/3$ through the Gamma:

$$Gamma(\theta|1,3) = Expon(\theta|1/3)$$

\textbf{d. What is the posterior $p(\theta|D,\hat{\lambda})$?}

\begin{align}
    p(\theta|D,\hat{\lambda}) & = p(D|\theta,\hat{\lambda})p(\hat{\lambda}) \\
    & = \prod_{i=1}^{N} \theta e^{-\theta x_i} \theta e^{-\theta \lambda} \\
    & = \prod_{i=1}^{N} \theta^2 e^{-\theta (x_i + \lambda)} \\
    & = \theta^{2N} \prod_{i=1}^{N} e^{-\theta \lambda -\theta x_i} \\
    & = \theta^{2N} e^{-\theta (\lambda + \sum_{i=1}^{N} x_i)} \\
    & = Gamma(\theta|2N,\lambda + \sum_{i=1}^{N} x_i)
\end{align}

\textbf{e. Is the exponential prior conjugate to the exponential likelihood?}

Yes, both the prior and the likelihood are of the Gamma distribution (remember
the exponential distribution is a special case of the Gamma distribution).

\textbf{f. What is the posterior mean, $E[\theta|D, \hat{\lambda}]$?}

The posterior is a Gamma as shown above, which has a mean of $a/b$.

\textbf{g. Explain why the MLE and posterior mean differ. Which is more
reasonable in this example?}

Since the posterior comes from an informative prior ($\hat{\lambda}$), the
posterior and the MLE will be different, but equal as $N \rightarrow \infty$.

In this example, the posterior is more reasonable, since the prior is more
informative.

}

\exercise[]{
\textbf{The book discussed using a Beta prior for a Bayesian inference of a
Bernoulli rate parameter. \\
a. Now consider the following prior, that believes the coin is fair, or is
slightly biased towards tails:}

\begin{align}
    p(\theta) & = \left\{\begin{matrix}
        0.5 & if\, \theta=0.5 \\ 
        0.5 & if\, \theta=0.4 \\ 
        0 & otherwise
    \end{matrix}\right. \\
    & = 0.5I(\theta - 0.5 = 0) + 0.5I(\theta - 0.4 = 0)
\end{align}

\textbf{Derive the MAP estimate under this prior as a function of $N_1$ and $N$.}

The posterior is given by

\begin{align}
    p(\theta|D) & = p(D|\theta)p(\theta) \\
    & = \theta^{N_1}(1-\theta)^{N_0}p(\theta) \\
    & = \theta^{N_1}(1-\theta)^{N_0}(0.5I(\theta-0.5=0) + 0.5I(\theta-0.4=0)) \\
    & = 0.5^{N_1+N_0+1}I(\theta-0.5=0) +
        0.5(0.4^{N_1})(0.6^{N_0})I(\theta-0.4=0)
\end{align}

Note that the prior is so restrictive that the likelihood is $0$ for all
$\theta$ except for $0.4$ and $0.5$. Thus, we can actually compute the
likelihood for both of these values of $\theta$ and find which one maximizes
the likelihood.

So, for each value of $\theta$, the posterior is

\begin{align}
    p(0.4|D) & \propto (0.4^{N_1})(0.6^{N_0}) \\
    p(0.5|D) & \propto 0.5^{N_1+N_0}
\end{align}

These are functions of $N_0$ and $N_1$, and the value of $\theta$ that
maximizes the posterior will depend of these. We can find these constraints
by calling one the MAP and seeing the requirements needed for $N_1$ and $N_0$.
Let's say that $\theta = 0.4$:

\begin{align}
    (0.4^{N_1})(0.6^{N_0}) & \geq 0.5^{N_1+N_0} \\
    N_1log(0.4) + N_0log(0.6) & \geq (N_1+N_0)log(0.5) \\
    N_1(log(0.4) - log(0.5)) & \geq N_0(log(0.5) - log(0.6)) \\
    N_1log(\frac{4}{5}) & \geq N_0log(\frac{5}{6}) \\
    N_1 & \geq \frac{log(5/6)}{log(4/5)} N_0 \\
    & \approx 0.8171 N_0
\end{align}

Thus, when $N_1 \geq 0.8171 N_0$, then $\theta_{MAP} = 0.4$, otherwise
$\theta_{MAP} = 0.5$.

\textbf{b. Suppose the true parameter is $\theta = 0.41$. Which prior leads
to a better estimate when $N$ is small? Which prior leads to a better estimate
when $N$ is large?}

Note the "other" prior in this is when you use a $Beta(\theta|\alpha,\beta)$
prior, which leads to the MAP

$$\hat{\theta} = \frac{N_1+\alpha-1}{N_1+N_0+\alpha+\beta-2}$$

With small datasets, the prior can overwhelm the posterior. Thus, your choice
of Beta could greatly influence the posterior in small datasets. For the
handmade prior above, the worst that could happen is $\theta=0.5$, which
results in small error, whereas you could have worse error using a Beta prior.

For large datasets, note that the best you can do with the handmade prior is
$\theta=0.4$. When the true value is $0.41$, this is not bad error, but note
that using a conjugate prior with large datasets tends to the MLE solution,
which, with a large enough dataset can get arbitrarily precise.

}

\exercise[]{
\textbf{Derive the posterior predictive distribution for a batch of data with
the dirichlet-multinomial model.}

Note that the predictive distribution for a single data point is given by

$$p(X=j|D) = \frac{\alpha_j+N_j}{\alpha_0+N}$$

Since the assumption is that all data points are i.i.d, we can use this as
a jumping off point:

\begin{align}
    p(\tilde{D}|D,\alpha) & = p(x_1|D,\alpha)p(x_2|D,\alpha,x_1)\cdots
                              p(x_n|D,\alpha,x_1,x_2,\cdots,x_{n-1}) \\
    & = \frac{\prod_{j=1}^{K}  \prod_{i=1}^{N_j^{new}-1} \alpha_j + N^{old}_j + i}
             {\prod_{i=1}^{N-1} \alpha + N^{old} + i} \\
    & = \frac{\prod_{j=1}^{K}(\alpha_j+N_j^{old}+N_j^{new}-1)!/(\alpha_j+N_j^{old})!}
             {(\alpha + N^{old} + N - 1)! / (\alpha + N^{old})!} \\
    & = \frac{\prod_{j=1}^{K}\Gamma(\alpha_j+N_j)/\Gamma(\alpha_j+N_j^{old})}
             {\Gamma(\alpha + N) / \Gamma(\alpha + N^{old})} \\
    & = \frac{\Gamma(\alpha + N^{old})}{\Gamma(\alpha + N)}
        \prod_{j=1}^{K} \frac{\Gamma(\alpha_j+N_j)}{\Gamma(\alpha_j+N_j^{old})}
\end{align}

}

\exercise[]{
\textbf{a. Suppose we compute the empirical distribution over letters of the
Roman alphabet plus the space character (a distribution over $27$ values) from
$2000$ samples. Suppose we see the letter "$e$" $260$ times. What is
$p(x_{2001} = e|D)$, if we assume $\theta \sim Dir(\alpha_1, ..., \alpha_{27})$,
where $\alpha_k = 10$ for all $k$?}

Recall that the posterior predictive of the Dirichlet-multinomial model is

$$p(X=j|D) = \frac{\alpha_j+N_j}{\alpha_0+N}$$

Given that $\alpha_k = 10$ for all $k$, this is simply

\begin{align}
    p(x_{2001} = e|D) & = \frac{10+260}{\sum_{k=1}^{K} \alpha_k + 2000} \\
    & = \frac{270}{2270} \approx 0.119
\end{align}

\textbf{b. Suppose, in the $2000$ samples, we saw "$e$" $260$ times, "$a$"
$100$ times, and "$p$" $87$ times. What is $p(x_{2001} = p, x_{2002} = a|D)$,
if we assume $\theta \sim Dir(\alpha_1,...,\alpha_{27})$, where $\alpha_k = 10$
for all $k$?}

Note that

$$p(x_{2001}=p,x_{2002}=a|D) = p(x_{2001}=p|D)p(x_{2002}=a|D)$$

since they are conditionally independent events. Using the same framework
as above, and letting $\alpha = \sum_{k=1}^{K} \alpha_k = 270$,

\begin{align}
    p(x_{2001}=p,x_{2002}=a|D) & = \frac{\alpha_p+N_p}{\alpha+N}
                                   \frac{\alpha_a+N_a}{\alpha+N} \\
    & = \frac{97\times 110}{(270+2000)^2} \\
    & \approx 0.0021
\end{align}

}

\exercise[]{
\textbf{Suppose $\theta \sim \beta(\alpha_1, \alpha_2)$, and we believe that
$E[\theta] = m$ and $var[\theta] = v$. Using Equation 2.62, solve for
$\alpha_1$ and $\alpha_2$ in terms of $m$ and $v$. What values do you get if
$m = 0.7$ and $v = 0.22$?}

Equation 2.62 states that

$$mean = \frac{a}{a+b}, mode=\frac{a-1}{a+b-2}, var=\frac{ab}{(a+b)^2(a+b+1)}$$

for a Beta distribution. Using these, we get a system of equations

\begin{align}
    m & = \frac{a}{a+b} \\
    m(a+b) & = a \\
    mb & = a(1 - m) \\
    b & = \frac{a(1-m)}{m}
\end{align}

By plugging this into the variance function above, it can be shown that

$$a = m \left (\frac{m(1-m)}{v} - 1 \right )$$

which you can then plug back into the equation for $b$ above to get

$$b = (1-m) \left (\frac{m(1-m}{v} - 1 \right )$$

If $m=0.7$ and $v=0.2^2=0.04$, then

$$a = 0.7 \left (\frac{0.7(1-0.7)}{0.04} - 1 \right ) = 2.975$$

and

$$b = (1-0.7) \left (\frac{0.7(1-0.7)}{0.04} - 1 \right ) = 1.275$$

}

\exercise[]{
\textbf{Suppose $\theta \sim \beta(\alpha_1, \alpha_2)$ and we believe that
$E[\theta] = m$ and $p(l<\theta<u)=0.95$. Write a program that can solve for
$\alpha_1$ and $\alpha_2$ in terms of $m$,  and $u$.}

We know the mean, so we can write

\begin{align}
    m & = \frac{\alpha_1}{\alpha_1+\alpha_2} \\
    m(\alpha_1+\alpha_2) & = \alpha_1 \\
    m\alpha_2 & = \alpha_1 - m\alpha_1 \\
    \alpha_2 & = \frac{\alpha_1}{m} - \alpha_1
\end{align}

We are given the quantiles, which we can express as

\begin{align}
\int_l^u \frac{1}{B(\alpha_1,\alpha_2)}\theta^{\alpha_1-1}(1-\theta)^{\alpha_2-1}d\theta
    = I_{u}(\alpha_1,\alpha_2) - I_{l}(\alpha_1,\alpha_2)
\end{align}

where $I_{x}(\alpha_1,\alpha_2) = \int_0^x Beta(\alpha_1,\alpha_2)$ is the
regularized incomplete beta function. We can minimize the squared discrepancy
between this and $0.95$:

\begin{align}
    0 & = \frac{d}{d\theta} [(
        I_{u}(\alpha_1,\alpha_2)d\theta
      - I_{l}(\alpha_1,\alpha_2) d\theta - 0.95 )^2] \\
    0 & = I_{u}(\alpha_1,\alpha_2) - I_{l}(\alpha_1,\alpha_2) - 0.95
\end{align}

Since this exercise involves writing a program, the code for this program is
found in an IPython notebook in this same directory.

}

\exercise[]{
\textbf{Suppose we toss a coin $N$ times and observe $N_1$ heads. Let
$N_1 \sim Bin(N,\theta)$ and $\theta \sim Beta(1, 1)$. Show that the marginal
likelihood is $p(N_1|N)=1/(N + 1)$.}

The key here is that $N_1$ and $N$ are sufficient statistics. Remember that
the posterior of the Beta-Binomial model is given by

\begin{align}
    p(\theta|D) & = \frac{p(D|\theta)p(\theta)}{p(D)} \\
                & = Beta(\theta|N_1+a,N_0+b) \\
    p(D) & = \frac{p(D|\theta)p(\theta)}{Beta(\theta|N_1+a,N_0+b)}
\end{align}

The marginal likelihood is given by

\begin{align}
    p(N_1|N) & = \int_{\theta} \frac{p(N_1|\theta,N)p(\theta|N)}{p(N_1,N)} d\theta \\
             & = \int_{\theta} \frac{Bin(N_1|\theta, N)Beta(\theta|1,1)}
                      {Beta(\theta|N_1+1,N_0+1)} d\theta
\end{align}

Since we are marginalizing over $\theta$, we can rewrite these using Beta
functions, like

\begin{align}
    p(N_1|N) & = \binom{N}{N_1} \frac{B(N_1+1, N-N_1+1)}
                  {B(N_1+1,N-N_1+1)/B(N_1,N-N_1)} \\
    & = \binom{N}{N_1} \frac{B(N_1+1,N-N_1+1)}{B(1,1)} \\
    & = \binom{N}{N_1} \frac{\Gamma(N_1+1)\Gamma(N-N_1+1)}{\Gamma(N+2)} \\
    & = \frac{N!}{N_1!(N-N_1)!} \frac{\Gamma(N_1+1)\Gamma(N-N_1+1)}{\Gamma(N+2)} \\
    & = \frac{N!N_1!(N-N_1)!}{N_1!(N-N_1)!(N+1)!} \\
    & = \frac{N!}{(N+1)N!} \\
    & = \frac{1}{N+1}
\end{align}

}

\exercise[]{
\textbf{Suppose we toss a coin $N=10$ times and observe $N_1=9$ heads. Let the
null hypothesis be that the coin is fair, and the alternative be that the coin
can have any bias, so $p(\theta)=Unif(0,1)$. Derive the Bayes factor $BF_{1,0}$
in favor of the biased coin hypothesis. What if $N=100$ and $N_1=90$?}

The Bayes factor is defined as

$$BF_{1,0} = \frac{p(D|alt)}{p(D|null)}$$

Let's look into the null hypothesis first. The null hypothesis says that the
coin is not biased, meaning $\theta=0.5$. Thus, the likelihood is

$$p(N_1|\theta=0.5) = \binom{N}{N_1} 0.5^{N_1}0.5^{N-N_1}= \binom{N}{N_1} 0.5^{N}$$

Note that the alternative hypothesis marginalizes across all $\theta$, and as
we saw in the last exercise, this is $\frac{1}{N+1}$.

So, the Bayes Factor is

\begin{align}
    BF_{1,0} & = \frac{1}{\binom{N}{N_1} (N+1)0.5^{N}} \\
             & = \frac{2^N}{\binom{N}{N_1} (N+1)}
\end{align}

Thus, is $N=10$ and $N_1=9$, then the Bayes Factor is $9.31$. This is
moderately strong evidence to accept the alternative.

If $N=100$ and $N_1=90$, then the Bayes Factor is $7.251 \times 10^{14}$. This
is very strong evidence to accept the alternative.

}

\exercise[]{
\textbf{This question sets up Naive Bayes as a linear classifier. \\
a. Write down an expression for the log posterior odds ratio, in terms of
the features and the parameters.}

The log posterior odds ratio is

\begin{align}
    log \frac{p(c=1|x_i)}{p(c=2|x_i)} & =
            log \frac{p(x_i|c=1,\theta)p(c=1)}{p(x_i|c=2,\theta)p(c=2)} \\
    & = log \frac{p(x_i|c=1,\theta)}{p(x_i|c=2,\theta)} \\
    & = log p(x_i|c=1,\theta) - log p(x_i|c=2, \theta) \\
    & = \phi (x_i)^{T} \beta_1 - \phi (x_i)^{T} \beta_2 \\
    & = \phi (x_i)^{T} (\beta_1 - \beta_2)
\end{align}

\textbf{b. Intuitively, words that occur in both classes are not very
"discriminative", and therefore should not affect our beliefs about the class
label. Consider a particular word $w$. State the conditions on $\theta_{1,w}$
and $\theta_{2,w}$ (or equivalently the conditions on $\beta_{1,w}$,
$\beta_{2,w}$) under which the presence or absence of $w$ in a test document
will have no effect on the class posterior (such a word will be ignored by the
classifier). Hint: using your previous result, figure out when the posterior
odds ratio is $0.5/0.5$.}

For a word $w$ to have no effect on the posterior, the log posterior odds
should equal $1$. Since the model is linear, we can narrow this down to one
word. We also note that we are considering words that exist in both classes,
so $\phi (x_{i,w}) = 1$. So,

\begin{align}
    1 & = \phi (x_i)^{T} (\beta_1 - \beta_2) \\
    \beta_1 & = \beta_2 \\
    log\frac{\theta_{1,w}}{1-\theta_{1,w}} & =
    log\frac{\theta_{2,w}}{1-\theta_{2,w}} \\
    \frac{\theta_{1,w}}{1-\theta_{1,w}} & = \frac{\theta_{2,w}}{1-\theta_{2,w}} \\
    \theta_{1,w}(1-\theta_{2,w}) & = \theta_{2,w}(1-\theta_{1,w}) \\
    \theta_{1,w}-\theta_{1,2}\theta_{2,w}&=\theta_{2,w}-\theta_{1,w}\theta_{2,w} \\
    \theta_{1,w} & = \theta_{2,w}
\end{align}

\textbf{c. Let there be $n_1$ documents of class $1$ and $n_2$ be the number of
documents in class $2$, where $n_1 = n_2$ (since e.g., we get much more
non-spam than spam; this is an example of class imbalance). If we use the above
estimate for $\theta_{c,w}$, will word w be ignored by our classifier? Explain
why or why not.}

Since the word is in all documents, then the given estimates
$\hat{\theta}_{cw}$ are

\begin{align}
    \hat{\theta}_{1w} & = \frac{1+n_1}{2+n_1} \\
    \hat{\theta}_{2w} & = \frac{1+n_2}{2+n_2}
\end{align}

Since $n_1 \neq n_2$, these quantities are not equal. We saw in part (b) that
the necessary requirement for the model to ignore a word is for
$\theta_{1w} = \theta_{2w}$, so we can be sure that the model will not ignore
this word.

\textbf{d. What other ways can you think of which encourage "irrelevant"
words to be ignored?}

Weighting each word by frequency using TF-IDF for example.

}

\exercise[]{
\textbf{a. How would you specify a "full" model that doesn't use Naive Bayes
assumption? How many parameters would it have?}

The Naive Bayes assumption allows for significant simplification in the model
specification. Without it, the best you could do is to use the chain-rule of
probability:

$$p(x_{1:D}|y=c) = p(x_1|y=c)p(x_2|x_1,y=c)\cdots p(x_D|x_1,...,x_{D-1},y=c)$$

The Naive Bayes assumption allows us to trim down the contingency table to
a workable amount. Since the features are binary, the number of parameters
in the full model is $2^D$.

\textbf{b. Assume the number of features $D$ is fixed. Let there be $N$
training cases. If the sample size $N$ is very small, which model (naive Bayes
or full) is likely to give lower test set error, and why?}

The number of parameters in the naive Bayes model is $DC$ vs. $2^D$ for the
full model. So, if $N$ is very small while $D$ remains fixed, it is very likely
that the full model will be over-parameterized and overfit on the training set.
Therefore, in this case, the naive Bayes model will perform better on the test
set, since it avoids the curse of dimensionality.

\textbf{c. What if the sample size $N$ was very large?}

In this case, the conditional independence assumption that the naive Bayes
model uses may be too restrictive to capture the patterns in the data, and
therefore the full model will likely perform better.

\textbf{d. What is the computational complexity of fitting the full and naive
Bayes model as a function of $N$ and $D$?}

Both of the models are $O(ND)$ worst case, since we can assume that it takes
$O(D)$ time to convert a bit array to an array index.

\textbf{e. What is the computational complexity at test time for the full
and naive Bayes model?}

The complexity for naive Bayes at test time is $O(CD)$. For the full model,
we loop through the classes and lookup the joint probability to use as the
prediction. So, the complexity is $O(CD)$. Note that in the full model, $D$
is a much larger number than in naive Bayes.

\textbf{f. Suppose the test case has missing data. Let $x_v$ be the visible
features of size $v$, and $x_h$ be the hidden (missing) features of size $h$,
where $v + h = D$. What is the computational complexity of computing
$p(y|x_v,\hat{\theta})$ for the full and naive Bayes models, as a function
of $v$ and $h$?}

The naive Bayes model could just skip over the missing features, so therefore
the complexity would still be $O(CD)$. The full model, however, would have to
create entries for all possible combinations of missing and non-missing
features, which would be $O(2^hD)$.

}

\exercise[]{
\textbf{Derive equation 3.76}

\begin{align}
    I(X,Y) & = \sum_{x_j} \sum_y p(x_j,y) log \frac{p(x_j,y)}{p(x_j)p(y)} \\
           & = \sum_{x_j} \sum_y p(x_j|y)p(y) log \frac{p(x_j|y)p(y)}{p(x_j)p(y)} \\
           & = \sum_{x_j} \sum_y p(x_j|y)p(y) log \frac{p(x_j|y)}{p(x_j)}
\end{align}

Since it is given that the features are binary, we can expand the summation:

\begin{align}
    I(X,Y) & = \sum_{x_j} \sum_y p(x_j|y)p(y) log \frac{p(x_j|y)}{p(x_j)} \\
           & = \sum_y p(x_j|y)p(y)log\frac{p(x_j|y)}{p(x_j)} +
                      (1-p(x_j|y)p(y)log\frac{1-p(x_j|y)}{1-p(x_j)} \\
           & = \sum_y \theta_{jy} \pi_y log \frac{\theta_{jy}}{\theta_j} +
               (1-\theta_{jy})\pi_y log\frac{1-\theta_{jy}}{1-\theta_j}
\end{align}

}

\exerciseshere

\section{Gaussian Models}

\exercise[]{
\textbf{Let $X \sim U(−1, 1)$ and $Y = X^2$. Clearly $Y$ is dependent on $X$
(in fact, $Y$ is uniquely determined by $X$). However, show that
$\rho(X,Y) = 0$. Hint: if $X \sim U(a, b)$ then $E[X]=(a + b)/2$ and
$var[X]=(b − a)^2/12$.}

Let's plug things into the definition of correlation:

\begin{align}
    \rho(X,Y) & = \frac{cov(X,Y)}{\sigma_X\sigma_Y} \\
              & = \frac{E[XY] - E[X]E[Y]}{\sigma_X\sigma_Y} \\
              & = \frac{E[X^3] - E[X]E[X^2]}{\sigma_X\sigma_{X^2}}
\end{align}

Note that to show this equals $0$, we just have to show that the numerator
is equal to $0$. To do this, we will compute each term:

$$E[X^3] = \frac{1}{2} \int_{-1}^{1} u^3p(u)du = 0$$

$$E[X^2] = \frac{1}{2} \int_{-1}^{1} u^2p(u)du = \frac{1}{3}$$

$$E[X] = \frac{-1 + 1}{2} = 0$$

So we have

\begin{align}
    \rho(X,Y) & = \frac{E[X^3] - E[X]E[X^2]}{\sigma_X\sigma_{X^2}} \\
              & = \frac{0 - 0\times \frac{1}{3}}{\sigma_X\sigma_{X^2}} \\
              & = 0
\end{align}

}

\exercise[]{
\textbf{Let $X \sim N(0,1)$ and $Y = WX$, where $p(W = −1) = p(W = 1) = 0.5$.
It is clear that $X$ and $Y$ are not independent, since $Y$ is a function of $X$. \\
a. Show that $Y \sim N(0,1)$.}

So, $W$ randomly changes the sign half the time on $X$. Thus,

$$Y \sim \frac{1}{2} N(0,1) - \frac{1}{2} N(0,1)$$

Let's write the distribution out for this:

\begin{align}
    p(Y|\mu,\sigma^2) & = \frac{1}{\sqrt{2\pi \sigma^2}}
                          e^{-\frac{(WX-\mu)^2}{2\sigma^2}} \\
    & = \frac{1}{2\sqrt{2\pi \sigma^2}}e^{-\frac{(-X-\mu)^2}{2\sigma^2}} +
        \frac{1}{2\sqrt{2\pi \sigma^2}}e^{-\frac{(X-\mu)^2}{2\sigma^2}} \\
    & = \frac{1}{2\sqrt{2\pi}}e^{-\frac{(-X)^2}{2}} +
        \frac{1}{2\sqrt{2\pi}}e^{-\frac{X^2}{2}} \\
    & = \frac{1}{2\sqrt{2\pi}}\left ( e^{-\frac{X^2}{2}} + e^{-\frac{X^2}{2}} \right ) \\
    & = \frac{1}{2\sqrt{2\pi}}\left ( 2e^{-\frac{X^2}{2}} \right ) \\
    & = \frac{1}{\sqrt{2\pi}}e^{-\frac{X^2}{2}} \\
    & = N(0,1)
\end{align}

\textbf{b. Show that $cov[X,Y]=0$.}

\begin{align}
    cov[X,Y] & = E[XY] - E[X]E[Y] \\
             & = E[E[XY|W]] - E[X]E[WX] \\
             & = \frac{1}{2}E[X^2] + \frac{1}{2}E[-X^2] - E[X]E[WX] \\
             & = \frac{1}{2}E[X^2] + \frac{1}{2}E[X^2] - E[X]E[WX] \\
             & = E[X^2] - E[X](\frac{1}{2}E[^2]+\frac{1}{2}E[-X^2]) \\
             & = E[X^2] - E[X^2] \\
             & = 0
\end{align}

}

\exercise[]{
\textbf{Prove that $-1 \leq \rho(X,Y) \leq 1$}

$$\rho(X,Y) = \frac{cov(X,Y)}{\sigma_X\sigma_Y}$$

This is trivial to prove with the Cauchy-Swartz inequality which states
that

$$|cov(X,Y)| \leq \sqrt{\sigma_X^2\sigma_Y^2}$$

because for $\rho(X,Y)$ to be $> 1$ or $< -1$, then
$|cov(X,Y)| > \sqrt{\sigma_X^2\sigma_Y^2}$, which is false.

}

\exercise[]{
\textbf{Show that, if $Y=aX+b$ for some parameters $a>0$ and $b$, then
$\rho(X,Y)=1$. Similarly show that if $a<0$, then $\rho(X,Y)=-1$.}

\begin{align}
    \rho(X,Y) & = \frac{cov(X,Y)}{\sigma_X\sigma_Y} \\
    & = \frac{E[(X-E[X])(Y-E[Y])]}{\sqrt{E[(X-E[X])^2]E[(Y-E[Y])^2]}} \\
\end{align}

Note that the quantity $(Y-E[Y])$ can be written as

\begin{align}
    Y-E[Y] & = aX+b-E[aX+b] \\
           & = aX+b-b-aE[X] \\
           & = a(X-E[X])
\end{align}

Plugging this, we get

\begin{align}
    \rho(X,Y) & = \frac{aE[(X-E[X])(X-E[X])]}{|a|\sqrt{E[(X-E[X])^2]E[(X-E[X])^2]}} \\
              & = \frac{E[(X-E[X])(X-E[X])]}{E[(X-E[X])^2]} \\
              & = \frac{E[(X-E[X])^2]}{E[(X-E[X])^2]} \\
              & = 1
\end{align}

If $a<0$, then this changes to

\begin{align}
    \rho(X,Y) & = \frac{aE[(X-E[X])^2]}{|a|\sqrt{E[(X-E[X])^2]E[(X-E[X])^2]}} \\
              & = -\frac{E[(X-E[X])^2]}{E[(X-E[X])^2]} \\
              & = -1
\end{align}

}

\exercise[]{
\textbf{Derive the normalization constant for multivariate Gaussian.}

We are trying to show that

$$(2\pi)^{D/2}|\Sigma|^{1/2} = \int exp(-\frac{1}{2}(x-\mu)^T\Sigma^{-1}(x-\mu))dx$$

Using eigenvalue decomposition on $\Sigma = U\Lambda U^T$, we can write this as

\begin{align}
    (2\pi)^{D/2}|\Sigma|^{1/2} & =
        \int exp(-\frac{1}{2}(x-\mu)^T U\Lambda^{-1} U^T(x-\mu))dx \\
    & = \int exp(-\frac{1}{2}u^T \Lambda^{-1} u) du \\
    & = \int exp(-\frac{1}{2} \sum_d \frac{u_d^2}{\lambda_d}) du \\
    & = \prod_{i=1}^D \int exp(-\frac{u_i^2}{2 \lambda_i}) du
\end{align}

Note that this is the product of single dimensional Gaussians. We know that
$\int exp(-\frac{u^2}{2\sigma^2}) = \sqrt{2\pi \sigma^2}$, and so we can rewrite
this expression as

\begin{align}
    \prod_{i=1}^D \int exp(-\frac{u_i^2}{2 \lambda_i}) du & =
        \prod_{i=1}^D \sqrt{2\pi \lambda_i} \\
    & = (2\pi)^{D/2} \prod_{i=1}^D \lambda_i^{1/2} \\
    & = (2\pi)^{D/2} |\Sigma|^{1/2}
\end{align}

}

\exercise[]{
\textbf{Derive the pdf of the bivariate Guassian with $\Sigma$ given.}

Note that

\begin{align}
    \Sigma^{-1} = 
    \frac{1}{\sigma_1^2 \sigma_2^2 - \rho^2 \sigma_1^2 \sigma_2^2}
    \begin{bmatrix}
        \sigma_2^2 & -\rho \sigma_1 \sigma_2 \\ 
        -\rho \sigma_1 \sigma_2 & \sigma_1^2 
    \end{bmatrix}
\end{align}

and

\begin{align}
    & (x-\mu)^T\Sigma^{-1}(x-\mu) = 
    \frac{1}{\sigma_1^2 \sigma_2^2 - \rho^2 \sigma_1^2 \sigma_2^2}
    \begin{bmatrix}
        x_1 - \mu_1 & x_2 - \mu_2
    \end{bmatrix}
    \begin{bmatrix}
        \sigma_2^2 & -\rho \sigma_1 \sigma_2 \\ 
        -\rho \sigma_1 \sigma_2 & \sigma_1^2 
    \end{bmatrix}
    \begin{bmatrix}
        x_1-\mu_1 \\ x_2-\mu_2 
    \end{bmatrix} \\
    & = \frac{1}{\sigma_1^2 \sigma_2^2 - \rho^2 \sigma_1^2 \sigma_2^2}
    \begin{bmatrix}
        \sigma_1^2(x_1-\mu_1) + \rho \sigma_1 \sigma_2 (x_2 - \mu_n) &
        \rho \sigma_1 \sigma_2 (x_1 - \mu_1) + \sigma_2^2 (x_2 - \mu_n)
    \end{bmatrix}
    \begin{bmatrix}
        x_1-\mu_1 \\ x_2-\mu_2 
    \end{bmatrix} \\
    & = \frac{1}{\sigma_1^2 \sigma_2^2 - \rho^2 \sigma_1^2 \sigma_2^2}
      (x_1 - \mu_1) (\sigma_1^2 (x_1 - \mu_1) + \rho \sigma_1 \sigma_2
         (x_2 - \mu_2)) + (x_2 - \mu_2) (\rho \sigma_1 \sigma_2
         (x_1 - \mu_1) + \sigma_2^2 (x_2 - \mu_2)) \\
    & = \frac{1}{\sigma_1^2 \sigma_2^2 - \rho^2 \sigma_1^2 \sigma_2^2}
    \sigma_1^2 (x_1-\mu_1)^2 + 2 \rho \sigma_1 \sigma_2(x_1-\mu_1)(x_2-\mu_2)
        + \sigma_2^2 (x_2-\mu_2)^2 \\
    & = \frac{1}{1-\rho^2}\frac{\sigma_1^2 (x_1-\mu_1)^2}{\sigma_1^2 \sigma_2^2} +
        \frac{2 \rho \sigma_1 \sigma_2(x_1-\mu_1)(x_2-\mu_2)}
             {\sigma_1^2 \sigma_2^2} + \frac{\sigma_2^2 (x_2-\mu_2)^2}
             {\sigma_1^2 \sigma_2^2} \\
    & = \frac{1}{1-\rho^2}\left ( \frac{(x_1-\mu_1)^2}{\sigma_1^2} +
        \frac{(x_2-\mu_2)^2}{\sigma_2^2} + 2\rho \frac{(x_1-\mu_1)}{\sigma_1}
        \frac{(x_2-\mu_2)}{\sigma_2} \right )
\end{align}

We see that this is the quantity requested of us in the exercise.

}

\exercise[]{
\textbf{Compute the conditional probability distribution of the given
bivariate Gaussian.}

Note that the conditional probability of two Gaussians is a Gaussian. Also
the conditional probability distribution is given by

\begin{align}
    p(x_1|x_2) & = N(x_1|\mu_{1|2},\Sigma_{1|2}) \\
    \mu_{1|2}  & = \mu_1 + \Sigma_{12}\Sigma_{22}^{-1}(x_2-\mu_2) \\
    \Sigma_{1|2} & = \Sigma_{11} - \Sigma_{12}\Sigma_{22}^{-1}\Sigma_{21}
\end{align}

It simplifies things greatly that we are considering a bivariate Gaussian,
since $\Sigma_{jk}$ becomes a scalar. Thus, after plugging in to the equations
given in the problem,

\begin{align}
    p(x_2|x_1) & = N(x_2|\mu_{2|1},\Sigma_{2|1}) \\
    \mu_{2|1} &=\mu_2+\sigma_1\sigma_2(\rho \frac{\sigma_2}{\sigma_1}(x_1-\mu_1)) \\
    & = \mu_2 + \rho \sigma_2^2 (x_1 - \mu_1) \\
    \Sigma_{2|1} & = \sigma_1\sigma_2 \frac{\sigma_2}{\sigma_1} -
                     \sigma_1\sigma_2 \rho^2 \frac{\sigma_2}{\sigma_1} \\
                 & = \sigma_2^2 + \rho^2 \sigma_2^2 \\
    p(x_2|x_1) & = N(x_2|\mu_2+\rho\sigma_2^2(x_1-\mu_1), \rho^2\sigma_2^2)
\end{align}

If $\sigma_2 = \sigma_1 = 1$, then

$$p(x_2|x_1) = N(x_2|\mu_2+\rho(x_1-\mu_1), \rho^2)$$

}

\exercise[]{
\textbf{This exercise is shown in the R notebook "ch4-8.ipynb"}}

\exerciseshere

\section{Bayesian Statistics}

\exercise[]{
\textbf{Prove that a mixture of conjugate priors is indeed conjugate.}

We are trying to show that

$$p(\theta|D) = \sum_k p(z=k|D)p(\theta|D,z=k)$$

with a prior of the form

$$p(\theta) = \sum_k p(z=k)p(\theta|z=k)$$

We know that the posterior can be given by

\begin{align}
    p(\theta|D) & = \frac{p(D|\theta)p(\theta)}{p(D)} \\
                & = \frac{p(D|\theta)\sum_k p(\theta|z=k)p(z=k)}{p(D)} \\
                & = p(D|\theta)\frac{\sum_k p(\theta|z=k)}{p(D)} \\
                & = \frac{\sum_k p(D|\theta,z=k)p(\theta|z=k)}{p(D)} \\
                & = \frac{\sum_k p(\theta,z=k|D)p(D)p(\theta|z=k)}
                         {p(D)\sum_k p(\theta,z=k)} \\
                & = \frac{\sum_k p(\theta,z=k|D)p(\theta|z=k)}
                         {\sum_k p(\theta|z=k)p(z=k)} \\
                & = \frac{\sum_k p(\theta,z=k|D)}{\sum_k p(z=k)} \\
                & = \sum_k p(\theta,z=k|D) \\
                & = \sum_k p(z=k|D)p(\theta|D,z=k)
\end{align}

}

\exercise[]{
\textbf{Optimal threshold on classification probability. \\
a. What is $\theta$ as a function of $\lambda_{01}$ and $\lambda_{10}$?}

The model can be expressed as

$$p(y|x) = 1 - p(y=0|x) = p(y=1|x) = p_1 = \hat{y}$$

The loss function can be expressed as

\begin{align}
    L(y,\hat{y}) & = y(1 - \hat{y})\lambda_{01} + \hat{y}(1 - y)\lambda_{10} \\
    L(0,\hat{y}) & = \hat{y}\lambda_{10} \\
    L(1,\hat{y}) & = (1-\hat{y})\lambda_{01}
\end{align}

The quantity we are trying to minimize is the expected loss function. This
is given by

\begin{align}
    E[L(y,\hat{y})] & = p_0L(0,\hat{y}) + p_1L(1,\hat{y}) \\
                    & = p_0\hat{y}\lambda_{10} + p_1\lambda_{01}(1-\hat{y})
\end{align}

We can take the derivate with respect to $\hat{y}$ and set this to zero, which
give us

\begin{align}
    0 & = \frac{d}{d\hat{y}}(p_0\lambda_{10}\hat{y} +
                             p_1\lambda_{01}-\hat{y}p_1\lambda_{01}) \\
      & = p_0\lambda_{10} - p_1\lambda_{01} \\
    p_1\lambda_{01} & = p_0\lambda_{10} \\
    p_1\lambda_{01} & = (1-p_1)\lambda_{10} \\
    p_1\lambda_{01} + p_1\lambda_{10} & = \lambda_{10} \\
    p_1(\lambda_{01} + \lambda_{10}) & = \lambda_{10} \\
    p_1 & = \frac{\lambda_{10}}{\lambda_{01} + \lambda_{10}}
\end{align}

Thus we see that the decision boundary occurs when
$p_1 = \frac{\lambda_{10}}{\lambda_{01} + \lambda_{10}}$. If $p_1$ is greater
than this quantity, we will classify this with class $1$, and $0$ otherwise.

\textbf{b. Show a loss matrix where the threshold is $0.1$.}

This is accomplished by plugging in to the quantity derived above:

\begin{align}
    0.1 & = \frac{\lambda_{10}}{\lambda_{01} + \lambda_{10}} \\
    0.1\lambda_{01} + 0.1\lambda_{10} & = \lambda_{10} \\
    0.1\lambda_{01} & = 0.9\lambda_{10} \\
    \lambda_{01} & = 9\lambda_{10}
\end{align}

Thus, a loss function of the following form will be sufficient:

$$\begin{bmatrix}0 & 9\lambda_{10} \\ \lambda_{10} & 0 \end{bmatrix}$$

}

\exercise[]{
\textbf{Reject option in classifiers. \\
a. Show that the minimum risk is obtained if decide $Y=j$ if
$p(Y=j|x)\geq p(Y=k|x)$ for all $k$ and
$p(Y=j|x)\geq 1-\frac{\lambda_r}{\lambda_s}$.}

Note that the first part of this has been shown above, namely that the most
likely class will be chosen. The second part, the reject option, will be shown
in this exercise.

The expected posterior loss in this case is given by

$$\rho(\hat{y}|x) = \sum_{k=1}^C p(y=k|x)L(y=k,\hat{y})$$

Let's say that the correct class is $j$. Then, the posterior loss is

\begin{align}
    \rho(\hat{y}|x) & = \sum_{k=1}^C p(y=k|x)L(y=k,\hat{y}) \\
    & = \sum_{k\neq j} p(y=k|x)\lambda_s \\
    & = (1 - p(y=j|x))\lambda_s
\end{align}

We need to see when this quantity is better than the reject option. If the
reject option is chosen, then the posterior loss is given by

$$\rho(\hat{y}|x) = \sum_{k=1}^C p(y=k|x)\lambda_r = \lambda_r$$

Thus for us not to choose the reject option,

\begin{align}
    (1 - p(y=j|x))\lambda_s & \geq \lambda_r \\
    1 - p(y=j|x) & \geq \frac{\lambda_r}{\lambda_s} \\
    1 - \frac{\lambda_r}{\lambda_s} & = p(y=j|x)
\end{align}

\textbf{b. Qualitatively describe what happens as $\lambda_r / \lambda_s$
increases from $0$ to $1$.}

Note that we will choose class $j$ if it is the most likely class and if

$$p(y=j|x) \geq 1 - \frac{\lambda_r}{\lambda_s}$$

When the quantity $\lambda_r / \lambda_s$ is $0$, this means that
$\lambda_r = 0$. Note that $\lambda_r$ is the risk of rejection, and when
this is $0$, we will choose to reject often. In fact, we will only choose not
to reject when $p(y=j|x) = 1$.

When the quantity is $1$, that means that the risk of rejection is infinitely
high. Thus, this is equivalent to a classifier with no reject option.

}

\exercise[]{
\textbf{Suppose it costs $\$10$ to misclassify and $\$3$ for a human to
manually classify. \\
a. Suppose $p(y=1|x) = 0.2$. Which decision minimizes expected loss?}

The expected loss function is given by

$$\rho(\hat{y}|x) = p_1L(1,\hat{y}) + (1-p_1)L(0,\hat{y})$$

We can use this to plug in each decision

\begin{align}
    \rho(0|x) & = (1-0.2)\times 10 = 8 \\
    \rho(1|x) & = 0.2\times 10 = 2 \\
    \rho(r|x) & = 3
\end{align}

Thus, we would class this to class $0$.

\textbf{b. Now suppose $p(y=1|x) = 0.4$.}

Similarly we will plug the numbers in

\begin{align}
    \rho(0|x) & = (1-0.4)\times 10 = 6 \\
    \rho(1|x) & = 0.4\times 10 = 4 \\
    \rho(r|x) & = 3
\end{align}

Thus in this case, we will choose the reject option.

\textbf{c. Show that there are thresholds $\theta_0$ and $\theta_1$ such that
if $p_1 \leq \theta_0$, we will classify to $0$,
$\theta_0 \leq p_1 \leq \theta_1$ we will classify as reject, and
$p_1 \geq \theta_1$ we will classify as $1$.}

Let's run through each decision

\begin{align}
    \rho(0|x) & = (1 - p_1)\times 10 = 10 - 10p_1 = \rho_0 \\
    \rho(1|x) & = p_1\times 10 = 10p_1 = \rho_1 \\
    \rho(r|x) & = 3 = \rho_r
\end{align}

Let's say the correct choice is class $0$. Then

\begin{align}
    & 10-10p_1\geq 10p_1\rightarrow 1-p_1 \geq p_1\rightarrow p_1\leq 0.5 \\
    & 10p_1\leq 3\rightarrow p_1\leq 0.3
\end{align}

Now if the correct class is $1$, then

\begin{align}
    & 10p_1\geq 10-10p_1\rightarrow p_1\geq 1-p_1\rightarrow p_1\geq0.5 \\
    & 10-10p_1\leq 3\rightarrow 1-p_1\leq 0.3\rightarrow p_1\geq 0.7
\end{align}

Thus, the thetas are $\theta_0 = 0.3$ and $\theta_1 = 0.7$.

}

\exercise[]{
\textbf{Newsvendor problem}

\begin{align}
    E_{\pi}(Q) & = \int_Q^{\infty} (P-C)Qf(D)dD + \int_0^Q (P-C)Df(D)
                - \int_0^Q C(Q-D)f(D)dD \\
    & = (P-C)Q\int_Q^{\infty} f(D)dD + (P-C)\int_0^Q Df(D) dD
        - CQ\int_0^Q f(D)dD + C\int_0^Q Df(D) dD \\
    & = (P-C)Q(1-F(Q)) + P\int_0^Q Df(D) dD - CQ\int_0^Q f(D) dD \\
    & = (P-C)Q(1-F(Q)) + P\int_0^Q Df(D) dD - CQF(Q) \\
    & = (P-C)Q - (P-C)QF(Q) + P\int_0^Q Df(D) dD - CQF(Q) \\
    & = (P-C)Q - PQF(Q) + CQF(Q) + P\int_0^Q Df(D) - CQF(Q) \\
    & = (P-C)Q - PQF(Q) + P\int_0^Q Df(D) \\
    & = (P-C)Q - PQF(Q) + PQF(Q) - P\int_0^Q F(D) dD \\
    & = (P-C)Q - P\int_0^Q F(D) dD
\end{align}

Now we will take the derivate wrt $Q$. We then see that

$$\frac{d}{dQ} (P-C)Q - P\int_0^Q F(D) dD = (P-C) - PF(Q)$$

By setting this quantity to $0$, we see that

\begin{align}
    0 & = (P-C) - PF(Q) \\
    PF(Q) & = (P-C) \\
    F(Q) & = \frac{P-C}{P}
\end{align}

}

\exercise[]{
\textbf{Let $B = p(D|H_1)/p(D|H_0)$ be the Bayes factor of model 1. Suppose we
plot two ROC curves, one computed by thresholding $B$, and the other computed
by thresholding $p(H_1|D)$. Will they be the same or different?}

If we threshold on $B$, we are saying that the decision rule is given by

$$I(f(x) > B) = I(f(x) > \frac{p(D|H_1)}{p(D|H_0)})$$

Compare this if we threshold with $p(H_1|D)$:

$$I(f(x) > p(H_1|D))$$

The domains are different for the two, but are they still
monotonically related? We note that as $B$ increases to $\infty$, we
favor $H_1$ more heavily. Thus it makes sense that as $B$ increases,
so should $p(H_1|D)$.

Similarly, as $B$ decreases towards $0$, so should $p(H_0|D)$ decrease as well.
Thus, the ROC curves should be the same.

}

\exercise[]{
\textbf{Bayes model averaging improves predictive accuracy.}

For this exercise, note the following properties of KL-divergence:

\begin{align}
    KL(q||p) & = -\sum_x p(x)log\,q(x) + p(x)log\,p(x) \\
             & = -\sum_x p(x)(log\,q(x) + log\,p(x)) \\
             & = -\sum_x p(x)log\,q(x)p(x)
\end{align}

The expectation of the loss function of the plugin approximation is given by

\begin{align}
    E[L(\Delta, p^m)] & = E[-log(p^m(\Delta))] \\
    & = E[-log(p(\Delta|m,D))] \\
    & = -\sum_{m\in M} p(\Delta|m,D)p(m|D) log(p(\Delta|m,D))
\end{align}

The loss function of the Bayes model averaging estimate is given by

\begin{align}
    E[L(\Delta, p^{BMA})] & = E[-log(p^{BMA}(\Delta))] \\
    & = -E[log(\sum_{m\in M}p(\Delta|m,D)p(m|D))] \\
    & = -\sum_{m\in M}p(\Delta|m,D)p(m|D)log(\sum_{m\in M}p(\Delta|m,D)p(m|D))
\end{align}

One way to show that the Bayes model averaging is superior is to show that the
difference between the two is $\geq 0$. So, if we take the difference,

\begin{align}
    E[L(\Delta, p^m)] - E[L(\Delta, p^{BMA})] & =
    -\sum_{m\in M} p(\Delta|m,D)p(m|D) log(p(\Delta|m,D)) \\
    & +\sum_{m\in M}p(\Delta|m,D)p(m|D)log(\sum_{m\in M}p(\Delta|m,D)p(m|D)) \\
    & = -\sum_{m\in M}p(\Delta|m,D)p(m|D)(log\,p(\Delta|m,D)
        +log\sum_{m\in M}p(\Delta|m,D)p(m|D)) \\
    & = -\sum_{m\in M}p(m|D) \sum_{m\in M}p(\Delta|m,D)log\,p(\Delta|m,D)
        \sum_{m\in M}p(\Delta|m,D)p(m|D) \\
    & = -\sum_{m\in M}p(\Delta|m,D)log\,p(\Delta|m,D)
        \sum_{m\in M}p(\Delta|m,D)p(m|D) \\
    & = KL(p(\Delta|m,D), \sum_{m\in M}p(\Delta|m,D)p(m|D)) \\
    & \geq 0
\end{align}

}

\exercise[]{
\textbf{MLE and model selection for a 2d discrete distribution. \\
a. Write down the joint probability distribution $p(x,y|\theta)$ as a
$2\times 2$ table, in terms of $\theta = (\theta_1, \theta_2)$}

The likelihood can be factorized as

$$p(x,y|\theta) = p(y|x,\theta_2)p(x|\theta_1) = p(y|x,\theta_2)\theta_1$$

Using the definition of $p(y|x,\theta_2)$ shown in the problem, this can
be shown to be

\begin{tabular}{ l | c r }
     & y=0 & y=1 \\
    \hline
    x=0 & $\theta_1\theta_2$ & $\theta_1-(1-\theta_2)$ \\
    x=1 & $\theta_1(1-\theta_2)$ & $\theta_1\theta_2$ \\
\end{tabular}

\textbf{b. With the given dataset, what is the MLE of $\theta_1$
and $\theta_2$?}

The MLE of $\theta_1$ is the proportion of $x$s that are $1$. This means
that $\theta_1^{MLE} = \frac{4}{7}$.

The MLE of $\theta_2$ is the proportion of times that $x$ and $y$ agree.
This means that $\theta_2^{MLE} = \frac{2}{7}$.

To compute the likelihood ($p(D|\hat{\theta},M_2)$), we can use the
$2\times 2$ table to compute them. This gives us

\begin{align}
    p(D|\hat{\theta},M_2) & = \prod_{i=1}^N p(x_i,y_i|\hat{\theta})
\end{align}

Plugging in the data given, this is

\begin{align}
    p(D|\hat{\theta},M_2) & = \theta_1^7\theta_2^4(1-\theta_2)^3 \\
                          & = \frac{4}{7}^7\frac{2}{7}^4(1-\frac{2}{7})^3
\end{align}

\textbf{c. Now consider a model with 4 parameters, representing
$p(x,y|\theta) = \theta_{x,y}$. What is the MLE of $\theta$? What is
$p(D|\hat{\theta},M_4)$ where $M_4$ denotes this 4-parameter model?}

This situation is similar. Each $\theta_{x,y}$ is essentially one of the
cells in the $2\times 2$ table. Thus the MLE of $\theta$ is given by

\begin{align}
    \hat{\theta}
    & = \begin{bmatrix} \frac{1-\sum x}{N}\frac{1-\sum y}{N}(1-p(x)) \\
                        \frac{1-\sum x}{N}\frac{\sum y}{N}(1-p(x)) \\
                        \frac{\sum x}{N}\frac{1-\sum y}{N}(1-p(x)) \\
                        \frac{\sum xy}{N}(1-p(x)) \end{bmatrix} \\
    & = \begin{bmatrix} (3/7)(4/7)(1/2) \\
                        (3/7)(3/7)(1/2) \\
                        (4/7)(4/7)(1/2) \\
                        (2/7)(1/2) \end{bmatrix} \\
    & = \begin{bmatrix} 12/98 \\
                        9/98 \\
                        16/98 \\
                        4/98 \end{bmatrix} \\
\end{align}

This needs to be normalized, and when it is it becomes

\begin{align}
    \hat{\theta}
    & = \begin{bmatrix} 0.2926829268 \\
                        0.2195121951 \\
                        0.3902439024 \\
                        0.09756097561 \end{bmatrix} \\
\end{align}

To compute the likelihood, we take the data, figure out which $\theta_{x,y}$
is relevant, and take the product. Thus

\begin{align}
    p(D|\hat{\theta},M_4) & = \theta_{0,0}^2\theta_{0,1}
                              \theta_{1,0}^2\theta_{1,1}^2 \\
    & = 0.2926829268^2\times 0.2195121951\times 0.3902439024\times 0.09756097561^2
\end{align}

\textbf{d. Which model would be picked using leave-one-out CV?}

This would be extremely tedious to calcualte for both models by hand, but we
note that the second model has more parameters and thus would likely fit better
(or at least as well). Thus, it is likely that the second model will be chosen
for CV.

\textbf{e. Compute the BIC for both models? Which model does it prefer?}

The BIC is given by

$$BIC(M, D) = p(D|\hat{\theta}) - \frac{dof(M)}{2}log N$$

Let's look at the model complexity penalization term. For the two parameter
model, this is given by

$$\frac{2}{2}log N = log 7 \approx 0.845$$

For the four parameter model, this is

$$\frac{4}{2}log N = 2log 7 \approx 1.690$$

Thus, for the four parameter model to be preferred, it would have to have
a higher likelihood of $\geq log 7$. This is very unlikely (and is in fact
not the case), so the BIC would prefer the simpler model.

}

\exercise[]{
\textbf{Prove that the posterior median is the optimal estimate under L1 loss.}

The median is given by

\begin{align}
    p(y < a | x) & = \int_{-\infty}^a p(y|x)
                   = \int_{a}^{\infty} p(y|x) = p(y \geq a | x) = 0.5
\end{align}

The posterior expected loss under L1 loss is given by

$$\rho(a|x) = E[abs(y-a)|x]$$

By taking the derivative wrt $a$, we see that

\begin{align}
    \frac{d}{da} \rho(a|x) & = \frac{d}{da} E[abs(y-a)|x] \\
    & = E[\frac{d}{da}abs(y-a)|x] \\
    & = E[\frac{a-y}{|a-y|}|x] \\
    & = \int \frac{a-y}{|a-y|} p(\frac{a-y}{|a-y|}|x) dx \\
    & = \int sgn(a-y) p(sgn(a-y)|x) dx
\end{align}

So, the quantity we are minimizing is the sign of the quantity $a-y$. Note
that $\forall a < y, sgn(a-y) = -1$ and $\forall a > y, sgn(a-y) = 1$. Thus
the $a$s above $y$ "cancel out" the $a$s below $y$. Thus, we want to maximize
the number of $a - y$ that cancel each other out. This will occur at the
median.

}

\exercise[]{
\textbf{If $L_{FN} = cL_{FP}$, show that we should pick $\hat{y} = 1$ iff
$p(y=1|x)/p(y=0|x) > \tau$, where $\tau = \frac{c}{1+c}$.}

The text here is misprinted. In the example they give, $c=2$, meaning that
false negatives are twice as bad as false positives. They then say that this
would cause the model to have a decision threshold of $2/3$, meaning that it
would err on the side of saying negative. Since false negatives are more
costly, this doesn't make sense.

\begin{align}
    \rho(\hat{y}=0|x) & = cL_{FP}p(y=1|x) \\
    \rho(\hat{y}=1|x) & =  L_{FP}p(y=0|x) \\
\end{align}

We should pick class $1$ iff

\begin{align}
    \rho(\hat{y}=0|x) & > \rho(\hat{y}=1|x) \\
    cL_{FP}p(y=1|x) & > L_{FP}p(y=0|x) \\
    cp(y=1|x) & > p(y=0|x) \\
    \frac{p(y=1|x)}{p(y=0|x)} & > \frac{1}{c}
\end{align}

The above equation makes more sense. Let $c=2$. This means that classifying
something wrongly as negative is twice as bad as classifying something
wrongly as positive. So, we should err on the side of classifying things as
positive.

In this case, we just have to show that the ratio is $> \frac{1}{2}$ before
we classify as positive. This is in line with intuition.

}

\exerciseshere

\section{Frequentist Statistics}

\exercise[]{
\textbf{Suppose we have a completely random dataset with $N_1$ examples of
class $1$, and $N_2$ examples of class $2$, where $N_1=N_2$. What is the
best misclassification rate any method can achieve? What is the estimated
misclassification rate of the same method using LOOCV?}

The misclassification rate is given by

\begin{align}
    \frac{1}{N}\sum_{i=1}^N I(\hat{y}_i \neq y_i) & =
    \frac{1}{N_1+N_2}\sum_{i=1}^N y(1-\hat{y})
\end{align}

Since the input $x$ tells us nothing about the output $y$, the best
classification rate we could possibly do is $\frac{1}{K}$, where $K$ is
the number of classes. Thus, in this case, the best we could do is
$\frac{1}{2}$. For LOOCV, we see that this is given by

$$R(m,D,N) = \frac{1}{N}\sum_{i=1}^N L(y_i,f_m^{-i}(x_i))$$

Since the dataset is random, $f_m^{-i}(x_i) = f_m(x_i)$. This is identical
to the equation above, which means that we would get the same answer from
LOOCV as we did the simple misclassification rate in this case.

}

\exercise[]{
\textbf{James Stein estimator for Gaussian means. \\
a. Find the ML-II estimate of $m_0$ and $\tau_0^2$.}

The two stage model is given by

$$Y_i|\theta_i \sim N(\theta_i, 500), \theta_i|\mu \sim N(m_0, \tau_0^2)$$

The quantity we must optimize is given by

\begin{align}
    p(D|\theta_i,\mu) & = \prod_{n=1}^N N(\theta_i, 500)N(m_0|\tau_0^2) \\
    log\,p(D|\theta_i,\mu) & = \sum_{n=1}^N log\,N(\theta_i,500)
    + \sum_{n=1}^N log\,N(m_0|tau_0^2) \\
    & \propto -\sum_{n=1}^N (y_i - \theta_i)^2 - \sum_{n=1}^N
    log\,\sqrt{2\pi\tau_0^2} - \frac{(y_i-m_0)^2}{2\tau_0^2}
\end{align}

To find the ML-II estimate, we maximize this quantity with respect to $m_0$:

\begin{align}
    \frac{d}{dm_0} log\,p(D|\theta_i,\mu) & =
    \frac{d}{dm_0} (-\sum_{n=1}^N \frac{(y_i-m_0)^2}{2\tau_0^2}) \\
    & = \frac{d}{dm_0} (-\sum_{n=1}^N \frac{1}{2\tau_0^2}(y_i^2 -2m_0y_i +m_0^2)) \\
    & = \frac{d}{dm_0} (\sum_{n=1}^N \frac{1}{2\tau_0^2}(2m_0y_i - m_0^2)) \\
    & = \frac{d}{dm_0} (\sum_{n=1}^N \frac{1}{\tau_0^2}m_0y_i) -
        \frac{N^2}{\tau_0^2}m_0 \\
    & = \frac{d}{dm_0} (\frac{N}{\tau_0^2} \sum_{n=1}^N m_0y_i) -
        \frac{N^2}{\tau_0^2}m_0 \\
    & = -\frac{N^2}{\tau_0^2}m_0 + \frac{N}{\tau_0^2}\sum_{n=1}^N y_i
\end{align}

Setting this equal to $0$, we get

\begin{align}
    0 & = -\frac{N^2}{\tau_0^2}m_0 + \frac{N}{\tau_0^2}\sum_{n=1}^N y_i \\
    \frac{N^2}{\tau_0^2}m_0 & = \frac{N}{\tau_0^2}\sum_{n=1}^N y_i \\
    Nm_0 & = \sum_{n=1}^N y_i \\
    m_0 & = \frac{1}{N} \sum_{n=1}^N y_i
\end{align}

Thus we see that the ML-II of $m_0$ is just the arithmetic mean of the dataset.

We can similarly do this analysis for $\tau_0^2$:

\begin{align}
    \frac{d}{d\tau_0^2} log\,p(D|\theta_i,\mu) & \propto
    \frac{d}{d\tau_0^2} (-\sum_{n=1}^N (y_i - \theta_i)^2 - \sum_{n=1}^N
    log\,\sqrt{2\pi\tau_0^2} - \frac{(y_i-m_0)^2}{2\tau_0^2}) \\
    & = \frac{d}{d\tau_0^2} (-Nlog\,\sqrt{2\pi\tau_0^2} - \frac{N}{2\tau_0^2}
        \sum_{n=1}^N (y_i-m_0)^2) \\
    & = \frac{d}{d\tau_0^2} (-Nlog\,\sqrt{2\pi\tau_0^2}) -
        \frac{N\sum_{n=1}^N (y_i-m_0)^2}{4\tau_0^4} \\
    & = -\frac{N}{2\tau_0^2} - \frac{N\sum_{n=1}^N (y_i-m_0)^2}{4\tau_0^4}
\end{align}

Again, setting this to $0$ gives us

\begin{align}
    0 & = -\frac{N}{2\tau_0^2} - \frac{N\sum_{n=1}^N (y_i-m_0)^2}{4\tau_0^4} \\
    \frac{N}{2\tau_0^2} & = \frac{N\sum_{n=1}^N (y_i-m_0)^2}{4\tau_0^4} \\
    \frac{2\tau_0^2}{4\tau_0^4} & = \frac{N}{N\sum_{n=1}^N (y_i-m_0)^2} \\
    \frac{1}{2\tau_0^2} & = \frac{1}{\sum_{n=1}^N (y_i-m_0)^2} \\
    \tau_0^2 & = \frac{1}{2}\sum_{n=1}^N (y_i-m_0)^2
\end{align}

\textbf{b. Find the posterior estimates of $E[\theta_i|y_i,m_0,\tau_0]$
and $var[\theta_i|y_i,m_0,\tau_0]$ for $i=1$.}

Note that as the data tends to $\infty$, the ML estimate coverges to the
expected value of the actual parameter. So, the posterior estimate of
$E[\theta_1|y_1,m_0,\tau_0]$ is given by

\begin{align}
    E[\theta_1|y_1,m_0,\tau_0] & = N(\frac{y_1}{N},\frac{1}{2}
    (y_1 - \frac{y_1}{N})^2) \\
    & = N(\frac{y_1}{N}, \frac{1}{2}(\frac{N-1}{N}y_1)^2) \\
    & = N(\frac{y_1}{N}, \frac{(N-1)^2}{2N^2}y_1^2)
\end{align}

Plugging in the data, this gives us

\begin{align}
    E[\theta_1|y_1,m_0,\tau_0] & = N(\frac{y_1}{N}, \frac{(N-1)^2}{2N^2}y_1^2) \\
    & = N(\frac{1505}{6}, \frac{25}{72}1505^2)
\end{align}

\textbf{c. Give a $95\%$ credible interval for $p(\theta_i|y_i,m_0,\tau_0)$
for $i=1$. Do you trust this interval?}

The $95\%$ confidence interval is given by $m_0\pm 1.96\frac{\sigma^2}{\sqrt{N}}$.
Plugging the numbers in, we get

\begin{align}
    m_0 \pm 1.96\frac{\sigma^2}{\sqrt{N}} & =
    \frac{1505}{6} \pm 1.96\frac{\frac{25}{72}1505^2}{\sqrt{6}}
\end{align}

This interval is very wide.

\textbf{d. What do you expect would happen to your estimates if $\sigma^2$
were much smaller (say $\sigma^2 = 1$)? You do not need to compute the
numerical answer; just briefly explain what would happen qualitatively, and why.}

If the $\sigma^2$ were much smaller, this would not affect the variance of
$\theta_i$, which is affected only by $m_0$ and $\tau_0^2$. Thus, this would
have no effect on the interval width.

}

\exercise[]{
\textbf{Show that $\hat{\sigma}^2 = \frac{1}{N}\sum_{n=1}^N (x_n-\hat{\mu})^2$
is a biased estimator of $\sigma_2$.}

To show this, we note that

\begin{align}
    E[\hat{\sigma^2}] & = E[\frac{1}{N}\sum_{n=1}^N (x_n-\hat{\mu})^2] \\
    & = \frac{1}{N}E[\sum_{n=1}^N x_n^2 - 2\sum_{n=1}^N x_n\hat{\mu}
        + \sum_{n=1}^N \hat{\mu}^2] \\
    & = \frac{1}{N}E[\sum_{n=1}^N x_n^2-2N\hat{\mu}^2+\sum_{n=1}^N \hat{\mu}^2] \\
    & = E[\frac{1}{N}\sum_{n=1}^N x_n^2] - 2E[\hat{\mu}^2] +
        E[\frac{1}{N}\sum_{n=1}^N \hat{\mu}^2] \\
    & = E[x^2] - 2E[\hat{\mu}^2] + E[\hat{\mu}^2] \\
    & = E[x^2] - E[\hat{\mu}^2]
\end{align}

Note that the definition of variance says that

$$\sigma^2 = E[x^2] - E[x]^2$$

So, if we define $\sigma_x^2 = E[x^2] - E[x]^2$ and
$\sigma_{\hat{\mu}}^2 = E[\hat{\mu}^2] - E[\hat{\mu}]^2$, then

\begin{align}
    \sigma_x^2 - \sigma_{\hat{\mu}}^2 & =
        (E[x^2]-E[x]^2)-(E[\hat{\mu}^2]-E[\hat{\mu}]^2) \\
        & = E[x^2]-E[\hat{\mu}]^2-E[x]^2+E[\hat{\mu}]^2 \\
        & = E[x^2] - E[\hat{\mu}]^2 \\
        & = E[\hat{\sigma^2}]
\end{align}

The last two lines is valid because $E[x]^2 = E[\hat{\mu}]^2$, since
$\hat{\mu}$ is an unbiased estimator.

Let's investigate the quantity $\sigma_{\hat{\mu}}^2$:

\begin{align}
    \sigma_{\hat{\mu}}^2 & = Var[\hat{\mu}] \\
    & = \frac{1}{N^2} Var[\sum_{n=1}^N x_n] \\
    & = \frac{1}{N^2} \sum_{n=1}^N Var[x] \\
    & = \frac{N}{N^2} Var[x] \\
    & = \frac{1}{N} Var[x] \\
    & = \frac{1}{N} \sigma_x^2
\end{align}

Therefore,

\begin{align}
    E[\hat{\sigma}^2] & = \sigma_x^2 - \frac{1}{N} \sigma_x^2 \\
    & = \frac{N-1}{N} \sigma_x^2
\end{align}

}

\exercise[]{
\textbf{Estimate $\sigma^2$ when $\mu$ is known.}

We saw before that $E[\hat{\sigma^2}] = \sigma_x^2 - \sigma_{\hat{\mu}}^2$.
If $\mu$ is known, then $\sigma_{\hat{\mu}}^2 = 0$. Thus,

$$E[\hat{\sigma^2}] = \sigma_x^2$$

which means the estimate is unbiased.

}

\exerciseshere

\section{Linear Regression}

\exercise[]{
\textbf{Behavior of training set error with increasing sample size.}

There are two sources of errors to be seen. One is measurement error, this is
the irreducible error given in the noise of the problem. The other is
approximation error, which is the error in finding the coefficients.

When the sample size is small, the training set error will be overly optimistic
of the measurement error. That is, the model may overfit and thus understate
this error source. As training size increases, either the model must become
much more complex (so as to continue overfitting), or the overfitting will
stop and the measurement error will converge to the measurement error of the
problem.

As the model gets more complex, for a given sample size the approximation error
will increase, since there are more parameters to estimate.

Let's combine the two intuitions. For a complex model with small sample size,
the measurement error will be understated and the approximation error will be
higher. As the sample size increases, the measurement error will tend towards
the problem definition and the approximation error will decrease. This will
cause the training set error to increase to a plateau as sample size increases.

}

\exercise[]{
\textbf{Compute the MLE for $W$ in the given data.}

The weight matrix $W$ is given by

$$W = (X^TX)^{-1}X^TY$$

We are given a single $x$ vector that we expand using basis functions into

$$X = \begin{bmatrix} \phi(0) \\ \phi(0) \\ \phi(0) \\
                      \phi(1) \\ \phi(1) \\ \phi(1) \end{bmatrix}
    = \begin{bmatrix} 1 & 0 \\ 1 & 0 \\ 1 & 0 \\
                      0 & 1 \\ 0 & 1 \\ 0 & 1 \end{bmatrix}$$

In order to compute the MLE, we need to compute the quantity $(X^TX)^{-1}$.
This is given by

\begin{align}
    (X^TX)^{-1} = \left ( \begin{bmatrix} 1 & 0 \\ 1 & 0 \\ 1 & 0 \\
                          0 & 1 \\ 0 & 1 \\ 0 & 1 \end{bmatrix}^T
                          \begin{bmatrix} 1 & 0 \\ 1 & 0 \\ 1 & 0 \\
                          0 & 1 \\ 0 & 1 \\ 0 & 1 \end{bmatrix} \right )^{-1}
                = \begin{bmatrix} 3 & 0 \\ 0 & 3 \end{bmatrix}^{-1}
            = \begin{bmatrix} \frac{1}{3} & 0 \\ 0 & \frac{1}{3} \end{bmatrix}
\end{align}

We use this quantity to compute the weight matrix:

\begin{align}
    W & = (X^TX)^{-1}X^TY \\
      & = \begin{bmatrix} \frac{1}{3} & 0 \\ 0 & \frac{1}{3} \end{bmatrix}
          \begin{bmatrix} 1 & 0 \\ 1 & 0 \\ 1 & 0 \\
                          0 & 1 \\ 0 & 1 \\ 0 & 1 \end{bmatrix}^T
          \begin{bmatrix} -1 & -1 \\ -1 & -2 \\ -2 & -1 \\
                           1 &  1 \\  1 &  2 \\  2 &  1 \end{bmatrix} \\
      & = \begin{bmatrix} \frac{1}{3} & 0 \\ 0 & \frac{1}{3} \end{bmatrix}
          \begin{bmatrix} -4 & -4 \\ 4 & 4 \end{bmatrix}
      & = \begin{bmatrix} -\frac{4}{3} & -\frac{4}{3} \\
                           \frac{4}{3} &  \frac{4}{3} \end{bmatrix}
\end{align}

}

\exercise[]{
\textbf{Derive the MLE for ridge regression if the input is mean centered.}

The formula for the model is given by

$$J(w,w_0) = (y - Xw - w_01)^T (y - Xw - w_01) + \lambda w^Tw$$

We must optimize this for both $w$ and $w_0$. The key to this is to understand
that the mean of a matrix can be given by

$$\bar{X} = (1^T1)^{-1}1^TX$$

since $(1^T1) = N$, which means $(1^T1)^{-1} = 1/N$, and $1^TX = \sum_i X$. So,
if $X$ is mean centered, then this quantity is $0$.

Let's optimize for $w_0$ first:

\begin{align}
    \frac{d}{w_0} J(w,w_0) & = \frac{d}{w_0} (y^Ty - y^TXw - y^Tw_01 - w^TX^Ty + \\
                  & w^TX^TXw + w^TX^Tw_01 - 1^Tw_0^Ty + 1^Tw_0^TXw +
                  1^Tw_0^Tw_01 + \lambda w^Tw) \\
    & = \frac{d}{w_0}(-y^Tw_01 + w^TX^Tw_01 -1^Tw_0^Ty + 1^Tw_0^TXw + 1^Tw_0^Tw_01) \\
    & = -1^Ty + w^TX^T - 1^Ty + w^TX^T + 2w_01^T1 \\
    & = -2^Ty + 2w^TX^T + 2w_01^T1
\end{align}

Setting this equal to $0$ we get

\begin{align}
    0 & = -2^Ty + 1^Tw^TX^T + 2w_01^T1 \\
      & = -1^Ty + 1^Tw^TX^T + w_01^T1 \\
    w_01^T1 & = 1^Ty + w^TX^T \\
    w_0 & = (1^T1)^{-1}1^Ty + (1^T1)^{-1}1^Tw^TX^T \\
        & = (1^T1)^{-1}1^Ty \\
        & = \bar{y}
\end{align}

Now let's look at $w$. Optimizing for this we get

\begin{align}
    \frac{d}{w} J(w,w_0) & = \frac{d}{w} (y^Ty - y^TXw - y^Tw_01 - w^TX^Ty + \\
                  & w^TX^TXw + w^TX^Tw_01 - 1^Tw_0^Ty + 1^Tw_0^TXw +
                  1^Tw_0^Tw_01 + \lambda w^Tw) \\
    & = \frac{d}{w} (-y^TXw - w^TX^Ty + w^TX^TXw + w^TX^Tw_01 +
                     1^Tw_0^TXw + \lambda w^Tw) \\
    & = -y^TX - y^TX + 2X^TXw + 1^Tw_0^TX + 1^Tw_0^TX + 2\lambda w \\
    & = -2y^TX + 2^Tw_0^TX + 2\lambda w + 2X^TXw \\
    & = -y^TX + 1^Tw_0^TX + \lambda w + X^TXw \\
    & = (X^TX + \lambda I)w - y^TX + 1^T\bar{y}^TX \\
    & = (X^TX + \lambda I)w - (y^T + 1^T\bar{y}^T)X \\
    & = (X^TX + \lambda I)w - (y^T + ||y||)X
\end{align}

Setting this equal to $0$ we get

\begin{align}
    0 & = (X^TX + \lambda I)w - (X^Ty + X^Tw_0)^T \\
    (X^TX + \lambda I)w & = (X^Ty + X^T(y^T - w^TX))^T \\
    (X^TX + \lambda I)w & = (X^Ty + X^Ty^T - X^Tw^TX)^T \\
    (X^TX + \lambda I)w & = (X^T(y + y^T - w^TX))^T
\end{align}

}

\exercise[]{
\textbf{Show that the MLE for the error variance in linear regression is
given by the empircal variance of the residual errors.}

Each residual is i.i.d and normal. Therefore, the likelihood is given by

\begin{align}
    L(y,\hat{y}) & = \prod_{i=1}^N N(y - w^Tx, \sigma^2) \\
    log\,L(y,\hat{y}) & = \sum_{i=1}^N log\,N(y-w^Tx,\sigma^2) \\
    & =-\sum_{i=1}^Nlog(\sqrt{2\pi\sigma^2})+\frac{(y_i-w^Tx_i)^2}{2\sigma^2} \\
    & =-Nlog(\sqrt{2\pi\sigma^2})+\sum_{i=1}^N \frac{(y_i-w^Tx_i)^2}{2\sigma^2}
\end{align}

Taking the derivative of this wrt $\sigma^2$, we get

\begin{align}
    \frac{d}{d\sigma^2} (-Nlog(\sqrt{2\pi\sigma^2}) +
           \sum_{i=1}^N \frac{(y_i-w^Tx_i)^2}{2\sigma^2})
    = -\frac{N}{2\sigma^2} + \frac{(y_i-w^Tx_i)^2}{2\sigma^4} \\
\end{align}

Setting this to zero, get get

\begin{align}
    0 & = -\frac{N}{2\sigma^2} + \frac{(y_i-w^Tx_i)^2}{2\sigma^4} \\
    \frac{N}{2\sigma^2} & = \frac{(y_i-w^Tx_i)^2}{2\sigma^4} \\
    2N\sigma^4 & = 2\sigma^2(y_i-w^Tx_i)^2 \\
    N\sigma^2 & = (y_i-w^Tx_i)^2 \\
    \sigma^2 & = \frac{1}{N} \sum_{i=1}^N (y_i-w^Tx_i)^2
\end{align}

}

\exercise[]{
\textbf{Derive the MLE for the offset term in linear regression.}

The loss function with the offset term separated is given by

$$L(y,\hat{y}) = \frac{1}{N} \sum_{i=1}^N (y-w^Tx_i-w_0)^T(y-w^Tx_i-w_0)$$

Taking the derivative of this with respect to $w_0$ we get

\begin{align}
    \frac{d}{dw_0} L(y,\hat{y}) & = \frac{1}{N} \sum_{i=1} y - w^Tx_i - w_0
\end{align}

Setting this to $0$ we see that

\begin{align}
    0 & = \frac{1}{N} \sum_{i=1}^N y - w^Tx_i - w_0 \\
    w_0 & = \frac{1}{N} \sum_{i=1}^N y - w^Tx_i \\
    & = \bar{y} - \bar{x}^Tw
\end{align}

If we similarly solve for $w$, then

\begin{align}
    \frac{d}{dw} L(y,\hat{y}) & = \frac{d}{dw} ((y-w^TX-w_0)^T(y-w^TX-w_0)) \\
    & = 2(y-w^TX-w_0)X^T \\
    & = (2y - 2w^TX - 2w_0)X^T \\
    & = 2X^Ty - 2w^TX^TX - 2w_0X^T
\end{align}

Setting this equal to $0$ we get

\begin{align}
    0 & = 2X^Ty - 2w^TX^TX - 2w_0X^T \\
    wX^TX & = X^Ty - w_0X^T \\
    w & = (X^TX)^{-1}X^Ty - (X^TX)^{-1}w_0X^T \\
      & = (X^TX)^{-1}X^Ty - (X^TX)^{-1}(\bar{y}-\bar{X}^Tw)X^T \\
      & = (X^TX)^{-1}X^Ty - (X^TX)^{-1}X^T\bar{y} + (X^TX)^{-1}\bar{X}^TwX^T \\
      & = (X^TX)^{-1}X^T(y-\bar{y}) + (X^TX)^{-1}\bar{X}^TwX^T \\
    w - (X^TX)^{-1}\bar{X}X^Tw & = (X^TX)^{-1}X^Ty_c \\
    w(1 - (X^TX)^{-1}\bar{X}X^T) & = (X^TX)^{-1}X^Ty_c \\
    w & = (1-(X^TX)^{-1}\bar{X}X^T)^{-1}(X^TX)^{-1}X^Ty_c \\
    & = (X^TX)^{-1}X^Ty_c - (\bar{X}X^T)^{-1}(X^TX)(X^TX)^{-1}X^Ty_c \\
    & = (X^TX)^{-1}X^Ty_c - (\bar{X}X^T)X^Ty_c \\
    & = ((X-\bar{X})^T(X-\bar{X}))^{-1}((X-\bar{X})^T)y_c \\
    & = (X_c^TX_c)^{-1}X_cy_c
\end{align}

}

\exercise[]{
\textbf{Derive the MLE for simple linear regression.}

Simple linear regression is just linear regression in the 1d case. The loss
function is given by

\begin{align}
    L(y,\hat{y}) & = \frac{1}{2} \sum_{i=1}^N (y_i - w_1x_i - w_0)^2
\end{align}

where we use $\frac{1}{2}$ instead of $\frac{1}{N}$ because this will be easier
to take the derivative of. Taking the derivative wrt $w_0$, we get

\begin{align}
    \frac{d}{dw_1} L(y,\hat{y}) & =
    \frac{d}{dw_1} \frac{1}{2} \sum_{i=1}^N (y_i - w_1x_i - w_0)^2 \\
    & = \sum_{i=1}^N y_i - w_1x_i - w_0
\end{align}

Setting this equal to $0$, we get

\begin{align}
    0 & = \sum_{i=1}^N y_i - w_1x_i - w_0 \\
    Nw_0 & = \sum_{i=1}^N y_i - w_1x_i \\
    w_0 & = \frac{1}{N} \sum_{i=1}^N y_i - \frac{1}{N}w_1 \sum_{i=1}^N x_i \\
    & = \bar{y} - w_1\bar{x}
\end{align}

Taking the derivative of the loss function wrt $w_1$, we get

\begin{align}
    \frac{d}{dw_1} L(y,\hat{y}) & =
    \frac{d}{dw_1} \frac{1}{2} \sum_{i=1}^N (y_i - w_1x_i - w_0)^2 \\
    & = \sum_{i=1}^N (y_i - w_1x_i - w_0)x_i
\end{align}

Setting this equal to $0$, we get

\begin{align}
    0 & = \sum_{i=1}^N (y_i - w_1x_i - w_0)x_i \\
    & = \sum_{i=1}^N y_ix_i - w_1x_i^2 - w_0x_i \\
    w_1\sum_{i=1}^N x_i^2 & = \sum_{i=1}^N x_iy_i - (\bar{y}-w_1\bar{x})x_i \\
    w_1\sum_{i=1}^N x_i^2 & = \sum_{i=1}^N x_iy_i - \bar{y}\sum_{i=1}^N x_i
                               + w_1\bar{x} \sum_{i=1}^N x_i \\
    w_1\sum_{i=1}^N x_i^2 & = \sum_{i=1}^N x_iy_i - N\bar{x}\bar{y}
                               + Nw_1\bar{x}^2 \\
                              + Nw_1\bar{x}^2 \\
    w_1\sum_{i=1}^N x_i^2 - Nw_1\bar{x}^2 & =\sum_{i=1}^N x_iy_i-N\bar{x}\bar{y} \\
    w_1(\sum_{i=1}^N x_i^2-N\bar{x}^2) & = \sum_{i=1}^N x_iy_i-N\bar{x}\bar{y} \\
    w_1 & = \frac{\sum_{i=1}^N x_iy_i - N\bar{x}\bar{y}}
                         {\sum_{i=1}^N x_i^2 - N\bar{x}^2}
\end{align}

}

\exercise[]{
\textbf{Sufficient statistics for online linear regression. \\
a. What are the minimal set of statistics that we need to estimate $w_1$?}

Using the definitions defined in the problem text, we see that

\begin{align}
    w_1 & = \frac{\sum_i (x_i - \bar{x})(y_i - \bar{y})}{\sum_i (x_i - \bar{x})^2}
          = \frac{NC_{xy}^{(n)}}{NC_{xx}^{(n)}}
          = \frac{C_{xy}^{(n)}}{C_{xx}^{(n)}}
\end{align}

\textbf{b. What are the minimal set of statistics that we need to estimate $w_0$?}

We can see that

\begin{align}
    w_0 & = \bar{y} - w_1\bar{x} = \bar{y}^{(n)} -
              \frac{C_{xy}^{(n)}}{C_{xx}^{(n)}}\bar{x}^{(n)}
\end{align}

\textbf{c. Derive equation for online updating $\bar{y}$.}

We see that

\begin{align}
    \bar{y}^{(n+1)} & = \frac{1}{n+1}\sum_{i=1}^{n+1} y_i \\
    & = \frac{1}{n+1}(n\bar{y} + y_{n+1}) \\
    & = \frac{n}{n+1}\bar{y} + \frac{1}{n+1}y_{n+1} \\
    & = \bar{y} - \frac{1}{n+1}\bar{y} + \frac{1}{n+1}y_{n+1} \\
    & = \bar{y} - \frac{1}{n+1}(\bar{y} + y_{n+1})
\end{align}

\textbf{d. Derive the update equation for $C_{xy}$}

\begin{align}
    C_{xy}^{(n+1)} & = \frac{1}{n+1}\sum_{i=1}^{n+1}
                       (x_i-\bar{x}^{(n+1)})(y_i-\bar{y}^{(n+1)}) \\
    & = \frac{1}{n+1}(x_{n+1}-\bar{x}^{(n+1)})(y_{n+1}-\bar{y}^{(n+1)}) +
        \frac{1}{n+1}\sum_{i=1}^n (x_i-\bar{x}^{(n+1)})(y_i-\bar{y}^{(n+1)}) \\
    & = \frac{1}{n+1}(x_{n+1}y_{n+1} - x_{n+1}\bar{y}^{(n+1)} -
                      y_{n+1}\bar{x}^{(n+1)} + \bar{x}^{(n+1)}\bar{y}^{(n+1)}) \\
    & + \frac{1}{n+1}\sum_{i=1}^n (x_iy_i-x_i\bar{y}^{(n+1)}-y_i\bar{x}^{(n+1)}
                                   -\bar{x}^{(n+1)}\bar{y}^{(n+1)})
\end{align}

By plugging in the equation we derived in part c., we can see that

\begin{align}
    C_{xy}^{(n+1)} & = \frac{1}{n+1}\left [
        x_{n+1}y_{n+1}+nC_{xy}^{(n)}+n\bar{x}^{(n)}\bar{y}^{(n)}
        -(n+1)\bar{x}^{(n+1)}\bar{y}^{(n+1)} \right ]
\end{align}

\textbf{Parts e. and f. can be found in the IPython notebook ch7-7.ipynb}

}

\exercise[]{
\textbf{Bayesian linear regression in 1d with known $\sigma^2$ \\
a. Compute unbiased estimate of $\hat{\sigma}^2$, using $w$ as the MLE.}

This is implemented in code in the IPython notebook ch7-8.ipynb.

\textbf{b. Assume $p(w) = p(w_0)p(w_1)$, and $p(w_0)$ is uniform and $p(w_1)$
is $N(0,1)$. What is $p(w)$?}

An uniformative uniform prior can be expressed as a normal with infinite
variance (or zero precision). Thus

\begin{align}
    p(w) & = N(0,\infty)N(0,1) \\
         & = N(\begin{bmatrix} 0 \\ 0 \end{bmatrix},
             \begin{bmatrix} \infty & 0 \\ 0 & 1 \end{bmatrix})
\end{align}

\textbf{c. Compute the marginal posterior of the slope $p(w_1|D,\sigma^2)$.
What is $E[w_1|D,\sigma^2]$ and $var[w_1|D,\sigma^2]$?}

Using the results from 7.6.1, we see that the posterior is given by

\begin{align}
    p(w|x,y,\sigma^2) & = N(w|w_N,V_N) \\
    w_N & = \frac{1}{\sigma^2}V_N\sum_{i=1}^N x_iy_i \\
    V_N & = \frac{\sigma^2}{\sigma^2 + \sum_{i=1}^N x_i^2}
\end{align}

By plugging in the data provided, we get

$$p(w|x,y,\sigma^2) = N(w|0.012567,0.000407)$$

These two parameters to the distribution are the expected value and the
variance, respectively,

\textbf{d. What is a $95\%$ credible interval for $w_1$?}

Since the posterior is Gaussian, we note that the $95\%$ credible interval is
given by $\mu \pm 1.96\sigma^2$. From the numbers given, this is

$$0.012567 \pm 1.96 \times 0.000407 = [0.01176928, 0.01336472]$$

}

\exerciseshere

\section{Logistic Regression}

\exercise[]{
\textbf{Spam classification using logistic regression.}

This exercise is written in the IPython notebook ch8-1-2.ipynb.

}

\exercise[]{
\textbf{Spam classification using Naive Bayes.}

This exercise is written in the IPython notebook ch8-1-2.ipynb.

}

\exercise[]{
\textbf{Gradient and Hessian of log-likelihood for multinomial logistic regression. \\
a. Derive the derivative of the sigmoid function.}

The sigmoid function is given by

$$\sigma(a) = \frac{1}{1+e^{-a}}$$

Taking the derivative of this, we get

\begin{align}
    \frac{d}{da} \sigma(a) & = \frac{d}{da} \frac{1}{1+e^{-a}} \\
    & = -\frac{1}{(1+e^{-a})^2} \frac{d}{da} (1+e^{-a}) \\
    & = -\frac{1}{(1+e^{-a})^2)} \times -e^{-a} \\
    & = \frac{e^{-a}}{(1+e^{-a})^2} \\
    & = e^{-a} \sigma(a)^2 \\
    & = (\frac{1}{\sigma(a)} - 1) \sigma(a)^2 \\
    & = \sigma(a)(1 - \sigma(a))
\end{align}

\textbf{b. Using the previous result and chain rule of calculus, derive an
expression for the gradient of the log likelihood.}

The NLL is given by

$$NLL(w) = \sum_{i=1}^N log(1 + exp(-\tilde{y}_iw^Tx_i))$$

The gradient of this is the derivative of this wrt $w$, which is

\begin{align}
    \frac{d}{dw} NLL(w) & = \frac{d}{dw} \sum_{i=1}^N
                            log(1 + exp(-\tilde{y}_iw^Tx_i)) \\
    & = \sum_{i=1}^N \sigma(\tilde{y}_iw^Tx_i)x_i\tilde{y}_i
\end{align}

We note that $\tilde{y} \in [-1, 1]$. Therefore, when $\tilde{y} = 1$ we get

$$\frac{d}{dw} NLL(w) = \sum_{i=1}^N \sigma(w^Tx_i)x_i = \sum_{i=1}^N \mu_ix_i$$

and when $\tilde{y} = -1$ we get

$$\frac{d}{dw} NLL(w) = -\sum_{i=1}^N \sigma(-w^Tx_i)x_i =
                        -\sum_{i=1}^N (1 - \sigma(w^Tx_i))x_i =
                        -\sum_{i=1}^N (1 - \mu_i)x_i$$

Therefore, using the fact that $y \in [0, 1]$, we can rewrite this as

$$\frac{d}{dw} NLL(w) = \sum_{i=1}^N (\mu_i - y_i)x_i$$

\textbf{c. Prove that the Hessian is positive definite.}

The Hessian is given by

$$H = X^TSX$$

where $S = diag(\mu_1(1-\mu_1), ..., \mu_n(1-\mu_n))$. Since we know that
$0 \geq \mu_i \leq 1$, we know that $S$ is positive. Therefore

\begin{align}
    H & = X^TSX \\
      & = tr(X^TSX) \\
      & = \sum_i \sum_j X_{ij} S_{ij} X_{ij} \\
      & = \sum_i \sum_j X_{ij}^2 S_{ij}
\end{align}

Since $X_{ij}^2 \geq 0 \, \forall X_{ij}$, then $H$ must be positive definite.

}

\exercise[]{
\textbf{Gradient and Hessian of log-likelihood for multinomial logistic regression. \\
a. Derive the Jacobian of the softmax.}

The softmax function is given by

$$S(\eta_i)_k = \frac{e^{\eta_{ik}}}{\sum_{j=1}^J e^{\eta_{ij}}}$$

Let's look at the derivative in the case where $\eta_{ij} = \eta_{ik}$. In
this case, the derivative is

\begin{align}
    \frac{\partial \mu_{ik}}{\partial \eta_{ij}} =
    \frac{\partial \mu_{ij}}{\partial \eta_{ij}} & =
    \frac{\partial}{\partial \eta_{ij}}
        \frac{e^{\eta_{ij}}}{\sum_{j=1}^J e^\eta_{ij}} \\
    & = \frac{e^{\eta_{ij}}}{\sum_{j=1}^J e^{\eta_{ij}}} -
    \frac{e^{2\eta_{ij}}}{(\sum_{j=1}^J e^{\eta_{ij}})^2}
\end{align}

Now let's look at the case when $\eta_{ij} \neq \eta_{ik}$. In this case, the
derivative is

\begin{align}
    \frac{\partial \mu_{ik}}{\partial \eta_{ij}} & =
    -\frac{e^{\eta_{ik}+\eta_{ij}}}{(\sum_{j=1}^J e^{\eta_{ij}})^2}
\end{align}

We can combine these into one equation using $\delta_{jk} = I(j = k)$ as follows:

\begin{align}
    \frac{\partial \mu_{ik}}{\partial \eta_{ij}} & =
    \delta_{jk} \left ( \frac{e^{\eta_{ij}}}{\sum_{j=1}^J e^{\eta_{ij}}} -
    \frac{e^{\eta_{ij}+\eta_{ik}}}{(\sum_{j=1}^J e^{\eta_{ij}})^2} \right )
    -(1-\delta_{jk})\frac{e^{\eta_{ik}+\eta_{ij}}}{(\sum_{j=1}^J e^{\eta_{ij}})^2} \\
    & = \delta_{jk}(\mu_{ij}-\mu_{ij}\mu_{ik}) - (1-\delta_{jk})\mu_{ij} \mu_{ik} \\
    & = \delta_{jk}\mu_{ij}-\delta_{jk}\mu_{ij}\mu_{ik}
      - \mu_{ij}\mu_{ik} + \delta_{jk}\mu_{ij}\mu_{ik} \\
    & = \delta_{jk}\mu_{ij} - \mu_{ij}\mu_{ik} \\
    & = \mu_{ij}(\delta_{jk} - \mu_{ik})
\end{align}

\textbf{b. Show that $\bigtriangledown_{w_c}l = \sum_i (y_{ic}-\mu_{ic})x_i$.}

The log likelihood is given by

\begin{align}
    l(W) & = \sum_{i=1}^N \left [ \left ( \sum_{c=1}^C y_{ic}w_c^Tx_i \right )
             - log \left ( \sum_{c'=1}^C exp(w_{c'}^Tx_i) \right ) \right ] \\
    & = \sum_{i=1}^N \sum_{c=1}^C y_{ic}w_c^Tx_i
      - \sum_{i=1}^N log \sum_{c'=1}^C exp(w_{c'}^Tx_i)
\end{align}

The likelihood for a particular class $c$ is given by

\begin{align}
    l(w_c) & = \sum_{i=1}^N y_{ic}w_c^Tx_i
             - \sum_{i=1}^N log \sum_{c'=1}^C exp(w_{c'}^Tx_i) \\
    & = w_c\sum_{i=1}^N y_{ic}x_i
      - \sum_{i=1}^N log \sum_{c'=1}^C exp(w_{c'}^Tx_i) \\
\end{align}

Taking the derivative of this wrt $w_c$, we get

\begin{align}
    \bigtriangledown_{w_c} l & = \sum_{i=1}^N y_{ic}x_i
    - \sum_{i=1}^N x_i\frac{e^{w_c^Tx_i}}{\sum_{c'=1}^C e^{w_{c'}^Tx_i}} \\
    & = \sum_{i=1}^N y_{ic}x_i - \mu_{ic}x_i \\
    & = \sum_{i=1}^N (y_{ic} - \mu_{ic})x_i
\end{align}

\textbf{c. Derive the block submatrix of the Hessian for the classes
$c$ and $c'$.}

We must take the derivative of the equation derived above.

\begin{align}
    \bigtriangledown_{w_c}^2 l = \bigtriangledown_{w_c} \bigtriangledown_{w_c} l
    & = \bigtriangledown_{w_c} \sum_{i=1}^N (y_{ic} - \mu_{ic})x_i \\
    & = -\sum_{i=1}^N \mu_{ic}(\delta_{cc'} - \mu_{ic'})x_i
\end{align}

}

\exercise[]{
\textbf{Symmetric version of $l_2$ regularized multinomial logistic regression.}

In this problem, we are asked to optimize a regularized multinomial model,
where the problem is overspecified (there are $C$ parameters and only $C-1$
degrees of freedom). In particular we are asked to optimize

\begin{align}
    \sum_{i=1}^N log\,p(y_i|x_i,W) - \lambda \sum_{c=1}^C w_c^Tw_c
    = \sum_{i=1}^N w_{c0} + w_c^Tx_i - log(\sum_{c=1}^C e^{w_{c0} + w_c^Tx_i})
      - \lambda \sum_{c=1}^C w_c^Tw_c
\end{align}

Using the definition of the gradient defined in previous problems, we can
derive the derivative of this optimizer wrt $w$ as

\begin{align}
    \frac{d}{w_c} \sum_{i=1}^N log\,p(y_i|x_i,W) - \lambda \sum_{c=1}^C w_c^Tw_c
    & = \sum_{i=1}^N (y_{ic} - \mu_{ic})x_i - 2\lambda w_c
\end{align}

Note that at the optimum, the gradient is necessarily $0$. Thus we have

\begin{align}
    0 & = -2\lambda w_c \\
    w_c & = \sum_{j=1}^J w_{cj} = 0
\end{align}

}

\exercise[]{
\textbf{Elementary properties of $l_2$ regularized logistic regression. \\
a. The cost function has multiple locally optimal solutions.}

This is False. For this to be false, the cost function would have to be convex.
That occurs when the Hessian is strictly positive. The likelihood is convex,
as is shown in the text, so the Hessian of the regularization term need be
strictly positive. The Hessian of this term is $\lambda$.

\textbf{b. Let $\hat{w}$ be global optimum. $\hat{w}$ is sparse?}

False. Regularization penalizes the weight vectors, but does not induce
sparsity. Regularization pushes the weight vectors towards zero, but
quadratically. Another way to think about this is that the curvature is
positive, and therefore will only tend to zero at the limits.

\textbf{c. If the training data is linearly separable, then some weights
$w_j$ might become infinite if $\lambda = 0$.}

This is true. If the data is linearly separable and there is no regularization,
then the weights can go to infinity, since they control the steepness of the
sigmoid curve. Regularization would prevent this, obvoiusly.

\textbf{d. The likelihood on the training set always increases as
we increase $\lambda$.}

False. As we increase $\lambda$, we restrict degrees of freedom. Thus we
incur bias on the training set for a reduction in variance.

\textbf{d. The likelihood on the test set always increases as
we increase $\lambda$.}

False. As we increase $\lambda$, we do in fact perform better on the test
set, but only to a point. At some point, the regularization term overwhelms,
and the model becomes too rigid to be useful even on the test set.

}

\exerciseshere

\section{Generalized linear models and the exponential family}

\exercise[]{
\textbf{Conjugate prior for univariate Gaussian in exponential family form.}

TODO

}

\exerciseshere

\usetikzlibrary{fit, positioning}

\section{Directed graphical models (Bayes nets)}

\exercise[]{
\textbf{Consider the DAG in Figure 10.14(a). Construct a new DAG where you
marginalize out $X$.}

The important thing to note here is that while conditioning a variable acts
like a ``blocker" node, marginalizing a node does the opposite. It removes
the node from the DAG, and adjusts all nodes accordingly.

To illustrate some properties of marginalization, let a shaded node mean a
marginalized node. In this case

\begin{tikzpicture}
\tikzstyle{main}=[circle, minimum size=10mm, thick, draw=black!80, node distance=16mm]
\tikzstyle{connect}=[-latex, thick]
\tikzstyle{box}=[rectangle, draw=black!100]
    \node[main, fill = white!100] (A) [label=right:A] { };
    \node[main, fill = white!100] (B) [right=of A, label=below:B] { };
    \node[main, fill = black!10]  (C) [below=of A, label=below:C] { };
    \path (A) edge [connect] (C)
          (B) edge [connect] (C);
\end{tikzpicture}

Node C is marginalized and this DAG reduces to

\begin{tikzpicture}
\tikzstyle{main}=[circle, minimum size=10mm, thick, draw=black!80, node distance=16mm]
\tikzstyle{connect}=[-latex, thick]
\tikzstyle{box}=[rectangle, draw=black!100]
    \node[main, fill = white!100] (A) [label=right:A] { };
    \node[main, fill = white!100] (B) [right=of A, label=right:B] { };
\end{tikzpicture}

In other words, since A and B are parents of C, marginalizing on C will make A
and B independent of each other. Another situation is

\begin{tikzpicture}
\tikzstyle{main}=[circle, minimum size=10mm, thick, draw=black!80, node distance=16mm]
\tikzstyle{connect}=[-latex, thick]
\tikzstyle{box}=[rectangle, draw=black!100]
    \node[main, fill = white!100] (A) [label=right:A] { };
    \node[main, fill = white!100] (B) [right=of A, label=below:B] { };
    \node[main, fill = black!10]  (C) [below=of A, label=below:C] { };
    \path (C) edge [connect] (A)
          (C) edge [connect] (B);
\end{tikzpicture}

Node C is marginalized and this DAG reduces to

\begin{tikzpicture}
\tikzstyle{main}=[circle, minimum size=10mm, thick, draw=black!80, node distance=16mm]
\tikzstyle{connect}=[-latex, thick]
\tikzstyle{box}=[rectangle, draw=black!100]
    \node[main, fill = white!100] (A) [label=below:A] { };
    \node[main, fill = white!100] (B) [right=of A, label=below:B] { };
    \path (A) edge [connect] (B)
          (B) edge [connect] (A);
\end{tikzpicture}

In this situation, C is a parent of A and B. Marginalizing out C causes and A
and B to be dependent to each other.

Applying these rules to the DAG in the problem text, we note that node X is
both a child and a parent node. By marginalizing X, it removes the dependence
on A and B, but adds dependencies to A and B through the children of X. In
other words, by removing X, now A and B influence E and F. Particularly,
both A and B influence both E and F. The DAG then looks like

\begin{tikzpicture}
\tikzstyle{main}=[circle, minimum size=10mm, thick, draw=black!80, node distance=16mm]
\tikzstyle{connect}=[-latex, thick]
\tikzstyle{box}=[rectangle, draw=black!100]
    \node[main, fill = white!100] (C) [] {C};
    \node[main, fill = white!100] (E) [below=of C] {E};
    \node[main, fill = white!100] (D) [right=of C] {D};
    \node[main, fill = white!100] (F) [below=of D] {F};
    \node[main, fill = white!100] (A) [above=of C] {A};
    \node[main, fill = white!100] (B) [above=of D] {B};
    \path (A) edge [connect] (F)
          (B) edge [connect] (E)
          (C) edge [connect] (E)
          (D) edge [connect] (F);
    \draw[bend right, ->] (A) to node [auto] {} (E);
    \draw[bend left, ->]  (B) to node [auto] {} (F);
\end{tikzpicture}

As you can see, both nodes A and B now affect E and F.

}

\exercise[]{
\textbf{Bayes ball. \\
a. Consider the DAG in Figure 10.14(b). List all variables that independent of
A given evidence on B.}

``Given evidence on B" is another way of saying ``conditional on B". To find
all of the variables that are independent of A, we look at all the paths that
each variable can arrive at A through, and examine them with the knowledge
of B.

Let's go through the nodes then.

Node C: this node is a parent of A, and thus is not independent of A, even
knowing B.

Node D: from node D, we can reach A through $\{D,B,A\}$, $\{D,G,E,C,A\}$, and
$\{D,G,I,H,C,A\}$. Conditioning on B blocks $\{D,B,A\}$, but not the others.
Therefore, D is not independent of A, conditioned on B.

Node E: we can reach node A using $\{E,C,A\}$, which is unblocked. Therefore
node E is not independent of node A.

Node F: we can reach node A using $\{F,C,A\}$, which is unblocked. Therefore
node F is not independent of node A.

Node G: we can reach node A using $\{G,E,C,A\}$, $\{G,D,B,A\}$, or
$\{G,I,H,F,C,A\}$. The path $\{G,D,B,A\}$ is blocked, but the others are not,
and therefore node G is reachable from node A.

Node H: we can reach node A using $\{H,F,C,A\}$, $\{H,I,G,E,C,A\}$, or
$\{H,I,G,D,B,A\}$. The path $\{H,I,G,D,B,A\}$ is blocked, but the others are
not, and therefore node H is reachable from node A.

Node I: we can reach node A using $\{I,H,F,C,A\}$, $\{I,G,D,B,A\}$, or
$\{I,G,E,C,A\}$. The path $\{I,G,D,B,A\}$ is blocked, but the others are not,
and therefore node I is reachable from node A.

Thus, the only node that is independent to A conditioning on B is B.

\textbf{b. Consider the DAG in Figure 10.14(c). List all variables that are
independent of A given evidence on J.}

Note that since G is the sole parent of J and J has no children, conditioning
on J is equivalent to conditioning on G.

Next, let's identify the hidden nodes. These are $\{H,I\}$. This means that
nodes ${C,F}$ are blocked from the rest of the DAG and are independent of A.

For the remaining nodes, note that conditioning on G opens up paths to A,
since all paths to A from other nodes that go through G will go through a
v-structure node. Thus, nodes $\{B,D,E\}$ are all not independent of A.

Thus, nodes $\{G,J,H,I,C,F\}$ are all independent of A after conditiong on J.

}

\exercise[]{
\textbf{Markov blanked for a DGM.}

We want to prove that the conditional for a node is given by the conditional
of the node with its parents times the conditional of its children with itself.
The conditional is given by

\begin{align}
    p(X_i|X_{-i}) & = \frac{p(X_{-i}|X_i)p(X_i)}{p(X_{-i})} \\
    & = \frac{p(X_i) \prod_{t \neq i}^T p(X_t|pa(X_t),X_i)}
             {\prod_{t \neq i}^T p(X_t|pa(X_t))}
\end{align}

Note that for the children $X_c$ of $X_i$, $pa(X_c) = X_i$. Thus, the numerator
becomes

\begin{align}
    p(X_i|X_{-i}) = \frac{p(X_i) \prod_{Y_j \in ch(X_i)} p(Y_j|pa(Y_j))
                                 \prod_{t \neq i}^T p(X_t|pa(X_t),X_i)}
                         {\prod_{t \neq i}^T p(X_t|pa(X_t))}
\end{align}

Note that Equation 10.7 comes in handy here. it says that
$p(X_{1:V}) = \prod_{t=1}^V p(x_t|pa(x_t))$. What follows from this is that
nodes are not affected by nodes that aren't its parents. Thus, we can write
this quantity as

\begin{align}
    p(X_i|X_{-i}) & = \frac{p(X_i|pa(X_i))\prod_{Y_j \in ch(X_i)}p(Y_j|pa(Y_j))
                                 \prod_{t \neq i}^T p(X_t|pa(X_t))}
                         {\prod_{t \neq i}^T p(X_t|pa(X_t))} \\
                  & = p(X_i|pa(X_i))\prod_{Y_j \in ch(X_i)}p(Y_j|pa(Y_j))
\end{align}

}

\exercise[]{
\textbf{Hidden variables in DGMs. \\
a. Assuming all nodes (including H) are binary and all CPDs are tabular,
prove that the model on the left has 17 free parameters.}

In general, a variable with $K$ states have $K-1$ free parameters. So, binary
variables will have $1$ free parameter. Thus $p(X_i)$ has $1$ free parameter.

We can write $P(X_n|X_{n-1},...,X_1)$ as
$P(X_n=n|X_{n-1}=m,...,X_1=a) = T_{nm...a}$. Since we are dealing with binary
variables, we note that $T_{nm...a}$ is indexed with a binary string of length
$n$. The number of states in a binary string of length $n$ is $2^{n-1}$. Thus,
$P(X_n|X_{n-1},...,X_1)$ has $2^{n-1}$ free parameters.

The joint distribution of this particular DGM is given by

\begin{align}
    p(X_{1:6}) & = p(X_1)p(X_2)p(X_3)\sum_h p(H=h|X_{1:3})p(X_4|H=h)
                   p(X_5|H=h)p(X_6|H=h)
\end{align}

Going left to right from each term in the joint, the number of free parameters
are given by

$$1 + 1 + 1 + 2^3 + 2 + 2 + 2 = 17$$

\textbf{b. Assuming all nodes are binary and all CPDs are tabular, prove that
the model on the right has 59 free parameters.}

The joint is given by

$$p(X_{1:6}) = p(X_1)p(X_2)p(X_3)p(X_4|X_{1:3})p(X_5|X_{1:4})p(X_6|X_{1:5})$$

Following the similar process above, we see that the number of free parameters
are given by

$$1 + 1 + 1 + 2^3 + 2^4 + 2^5 = 59$$

\textbf{c. Suppose we have a data set $D = X_{1:6}^n$ for $n = 1:N$, where we
observe the $X$s but not $H$, and we want to estimate the parameters of the
CPDs using maximum likelihood. For which model is this easier?}

Computing the MLE for DGM models means counting the proportion that fall onto
the entry of the CPD. Thus, with less free parameters, this is easier to
estimate.

For very large $N$, the more complex model is preferred, because it models
more interaction that may be useful. But for anything less than a very large
$N$, the simpler model is easier to estimate the CPDs using maximum likelihood.

}

\exercise[]{
\textbf{Bayes net for a rainy day.}
The joint distribution is given by

\begin{align}
    P(V,R,G,S) & = P(V)P(G)P(R|V,G)P(S|G)
\end{align}

\textbf{a. Write down an expression for $P(S=1|V=1)$ in terms of $\alpha$,
$\beta$, $\gamma$, and $\delta$.}

We note that $p(x_i) = p(x_i|pa(x_i))$. Thus,

\begin{align}
    P(S=1|V=1) & = P(S=1|V=1,G) = \sum_{g \in G} P(S=1|G=g)P(G=g) \\
               & = \alpha (1 - \gamma) + (1 - \alpha) (1 - \beta)
\end{align}

\textbf{b. Write down an expression for $P(S=1|V=0)$. Is this the same or
different?}

Since $V$ is not a predecessor of $S$, this expression would be the same as
$P(S=1|V=1)$.

\textbf{c. Find ML estimates of $\alpha$, $\beta$, and $\gamma$ using the
given dataset.}

ML estimates of a CPT is just the proportion of counts that fall into the
given cell of the table.

Since $\alpha = P(G=0)$, then $\alpha = \frac{1}{3}$. Since
$\beta = P(G=1|S=0)$, then $\beta = 0$. Since
$\gamma = P(G=0|S=0)$, then $\gamma = 1$.

Note that these ML estimate for $\beta$ falls into the zero-count problem.

}

\exercise[]{
\textbf{Fishing nets. \\
a. Classify the fish as salmon or sea bass.}

We first note that

\begin{align}
    p(X_2|X_1,X_4) & = p(X_1)p(X_2|X_1)p(X_4|X_2)
\end{align}

This much is true from exercise 10.3. Because we know the fish is thin, we
``select" this column from the $p(X_4|X_2)$ matrix, and then can formulate
this as a series of matrix multiplications:

\begin{align}
    p(X_2|X_1,X_4) & = \begin{bmatrix} 0.5 & 0 & 0 & 0.5 \end{bmatrix}
                       \begin{bmatrix} 0.9 & 0.1 \\  0.3 & 0.7 \\
                                       0.4 & 0.6 \\  0.8 & 0.2 \end{bmatrix}
                       \begin{bmatrix} 0.6 \\ 0.05 \end{bmatrix} \\
                   & = 0.38
\end{align}

Thus, there is a 38\% chance that the fish is a sea bass, and 62\% chance that
the fish is a salmon. So, we'd classify this as a salmon.

\textbf{Suppose all we know is that the fish is thin and medium lightness. What
season is it now, most likely?}

Now we are interested in predicting $p(X_1|X_3,X_4)$. This probability is
given by

\begin{align}
    p(X_1|X_3,X_4) & \propto p(X_3|X_2)p(X_4|X_2)p(X_2|X_1)p(X_1) \\
    & = (\begin{bmatrix} 0.33 & 0.1 \end{bmatrix} \otimes
         \begin{bmatrix} 0.6 & 0.05 \end{bmatrix})
        \begin{bmatrix} 0.9 & 0.1 \\  0.3 & 0.7 \\
                        0.4 & 0.6 \\  0.8 & 0.2 \end{bmatrix}^T
        \begin{bmatrix} 0.25 & 0.25 & 0.25 & 0.25 \end{bmatrix} \\
    & = \begin{bmatrix} 0.044675 & 0.015725 & 0.02055 & 0.03985 \end{bmatrix}
\end{align}

We can normalize this vector by dividing by its norm to get

$$p(X_1|X_3,X_4) = \begin{bmatrix} 0.3698262 & 0.1301738 &
                                   0.1701159 & 0.3298841 \end{bmatrix}$$

Thus, it is more likely that it is either fall or winter, with winter being
slightly more likely.

}

\exercise[]{
\textbf{Removing leaves in BN20 networks. \\
a. Show that we can safely remove all the hidden leaf nodes without affecting
the posterior over the disease nodes.}

This is a more informal proof. Note that this graph is directed, and thus
$z_{1:3}$ affects $x_{1:4}$, but not the other way around. Therefore, nodes
$x_3$ and $x_5$ will have no affect on the posterior $p(z_{1:3}|x_1,x_2,x_4)$,
and therefore this posterior can be modeled using either graphs.

\textbf{b. Show that we can analytically remove the leaves that are in the
``off state", by absorbing their effect into the prior of the parents.}

The posterior is given by

\begin{align}
    p(z_{1:d}|x_{i \in on}, x_{j \in off}) & = p(z_{1:d})
    \prod_{i \in on} p(x_i|pa(x_i)) \prod_{j \in off} p(x_j|pa(x_j)) \\
    & = p(z_{1:d}) \prod_{i \in on} p(x_i|pa(x_i))
                   \prod_{j \in off} p(x_j|pa(x_j)) \\
    & = p^*(z_{1:d}) \prod_{i \in on} p(x_i|pa(x_i))
\end{align}

where $p^*(z_{1:d}) = p(z_{1:d}) \prod_{j \in off} p(x_j|pa(x_j))$.

}

\exercise[]{
\textbf{Handling negative findings in the QMR network.}

Piggybacking off of the last exercise, we know that the absorption of the
negative findings is given by

\begin{align}
    p(z_{1:d}) \prod_{j \in f^{-}} p(x_j|pa(x_j)) =
    \prod_{d=1}^D p(z_d) \prod_{j \in f^{-}} p(x_j|pa(x_j))
\end{align}

We can see that there are $|D| \times |f^{-}|$ terms in this, and so therefore
this operation will take $O(|D||f^{-}|)$ time.

}

\exercise[]{
\textbf{Moralization does not introduce new independence statements.}

While moralization is not described in the text up to this point, the problem
text describes it. Consider a node $C$ with two parents, $A$ and $B$. Now
consider that we moralize $A$ and $B$ by adding an undirected edge connecting
them. This makes the joint distribution

$$p(A,B,C) = p(C|A,B)p(A,B)$$

as opposed to without moralization, which says the joint is given by

$$p(A,B,C) = p(C|A,B)p(A)p(B)$$

Essentially, moralization removes the assumption of independence between the
two parents. Thus, if they aren't independent, then moralization would reduce
the CI assumptions in the model. If the parents are in fact independent, then
$p(A,B)=p(A)p(B)$, and moralization has no effect.

}

\exerciseshere


\end{document}
