\section{Overview of Supervised Learning}

\exercise[]{
\textbf{Suppose each of $K$-classes has an associated target $t_k$, which is a
vector of all zeros, except a one in the $k$th position. Show that classifying
to the largest element of $\hat{y}$ amounts to choosing the closest target,
$min_k ||t_k - \hat{y}||$, if the elements of $\hat{y}$ sum to one.}

We are trying to show that

$$arg\,max_k\,\hat{y_k} = arg\,min_k ||t_k - \hat{y}||$$

To do this, we can show that for any $k^* \neq arg\,max_k\,\hat{y_k}$
and $k = arg\,max_k\,\hat{y_k}$,

$$||t_{k^*} - \hat{y}|| > ||t_{k} - \hat{y}||$$

Note that we can equivalently consider the $||\cdot||^2$ instead of $||\cdot||$,
because they are both monotonic for $\geq 0$.

Using the definition of the Euclidean norm,

\begin{align}
||t_{k^*} - \hat{y}||^2& = ||\hat{y}||^2 + ||t_{k^*}||^2 - 2t_{k^*}\hat{y} \\
                       & = \hat{y}^2 \\
\end{align}

Similarly,

\begin{align}
||t_k - \hat{y}||^2& = ||\hat{y}||^2 + ||t_k||^2 - 2t_k\hat{y} \\
                   & = \hat{y}^2 + 1 - 2\hat{y}
\end{align}

Therefore

\begin{align}
||t_{k^*} - \hat{y}||^2 - ||t_k - \hat{y}||^2& = \hat{y}^2 - (\hat{y}^2 + 1 - 2\hat{y}) \\
                 & = -1 + 2\hat{y} \\
                 & \geq 0
\end{align}

since $\hat{y}$ sums to one. So, $||t_k - \hat{y}||$ is minimized when
$k = arg\,max_k\,\hat{y_k}$.

}

\exercise[]{
Show how to compute the Bayes decision boundary for the simulation example
in Figure 2.5.
}

\exercise[]{
\textbf{Consider N data points uniformly distributed in a p-dimensional unit
ball centered at the origin. Suppose we consider a nearest-neighbor estimate
at the origin. Show that the median distance from the origin to the closest
data point is given by: $d(p, N) = (1 - \frac{1}{2}^{1/N})^{1/p}$.}

Let $m$ be the median distance from the origin to the closest point. This
means that the probability that all data points are further than $m$ is $0.5$.
"Further" simply means a greater norm. Since samples $x_i$ are i.i.d, we can
more formally state this as:

$$\prod_{i=1}^{N} P(||x_i|| > m) = \frac{1}{2}$$

Note that we can flip this around to be
$\prod_{i=1}^{N} P(||x_i|| \leq m) = \frac{1}{2}$.
Now we can use the cumulative function of the uniform distribution as follows:

\begin{align}
\prod_{i=1}^{N} P(||x_i|| \leq m)& = \prod_{i=1}^{N} 1 - ||m|| \\
                                 & = \prod_{i=1}^{N} 1 - m^p \\
                                 & = (1 - m^p)^N = \frac{1}{2}
\end{align}

Now we can solve for $m$:

\begin{align}
\frac{1}{2} & = (1 - m^p)^N \\
\frac{1}{2}^{1/N} & = 1 - m^p \\
m^p & = 1 - \frac{1}{2}^{1/N} \\
m & = (1 - \frac{1}{2}^{1/N})^{1/p}
\end{align}

}

\exercise[]{
\textbf{The edge effect problem discussed on page 23 is not peculiar to uniform
sampling from bounded domains. Consider inputs drawn from a spherical
multinormal distribution $X ∼ N(0, I_p)$. The squared distance from any sample
point to the origin has a $\chi^2_p$ distribution with mean $p$. Consider a
prediction point $x_0$ drawn from this distribution, and let $a = x_0/||x_0||$
be an associated unit vector. Let $z_i = a^Tx_i$ be the projection of each of
the training points on this direction. Show that the $z_i$ are distributed
$N(0, 1)$ with expected squared distance from the origin 1, while the target
point has expected squared distance $p$ from the origin. Hence for $p = 10$,
a randomly drawn test point is about 3.1 standard deviations from the origin,
while all the training points are on average one standard deviation along
direction $a$. So most prediction points see themselves as lying on the edge
of the training set.}

}

\exerciseshere
